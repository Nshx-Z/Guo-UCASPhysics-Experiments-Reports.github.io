%%%%%%%%%%%%%%%%%%%%%%%%%%%%%%%%%%%%%%%
%                                     %
%   %    %   %  %   %%%%%    %  %  %  %
%  %%   %%   %  %   %       %%  %  %  %
%   %    %   %%%%%  %%%%%    %  %%%%% %
%   %    %      %       %    %     %  %
%  %%%  %%%     %   %%%%%   %%%    %  %
%                                     %
%%%%%%%%%%%%%%%%%%%%%%%%%%%%%%%%%%%%%%%

%本实验报告由本人林诚皓和吉骏雄一起完成, 旨在方便LATEX原教旨主义者写实验报告, 避免Word文档因插入过多图造成卡顿. 

\documentclass[11pt]{article}

\usepackage[a4paper]{geometry}
\geometry{left=2.0cm,right=2.0cm,top=2.5cm,bottom=2.5cm}

\usepackage{ctex}
\usepackage{amsmath,amsfonts,graphicx,subfigure,amssymb,bm,amsthm}
\usepackage{algorithm,algorithmicx}
\usepackage[noend]{algpseudocode}
\usepackage{fancyhdr}
\usepackage{mathrsfs}
\usepackage{mathtools}
\usepackage[framemethod=TikZ]{mdframed}
\usepackage{fontspec}
\usepackage{adjustbox}
\usepackage{breqn}
\usepackage{fontsize}
\usepackage{tikz,xcolor}
\usepackage{multirow} 
\usepackage{booktabs}
\usepackage{tcolorbox}
\usepackage{pdfpages}
\usepackage{makecell}
\usepackage {framed}
% \usepackage{textcomp,mathcomp}
\setmainfont{Palatino Linotype}
\setCJKmainfont{SimHei}
\setCJKsansfont{Songti}
\setCJKmonofont{SimSun}
\punctstyle{kaiming}

\renewcommand{\emph}[1]{\begin{kaishu}#1\end{kaishu}}

%改这里可以修改实验报告表头的信息
\newcommand{\experiName}{弦上驻波及介质中声速测量}
\newcommand{\supervisor}{吴天涯}
\newcommand{\name}{李果}
\newcommand{\studentNum}{2022K8009906028}
\newcommand{\class}{01}
\newcommand{\group}{09}
\newcommand{\seat}{8}
\newcommand{\dateYear}{2023}
\newcommand{\dateMonth}{12}
\newcommand{\dateDay}{11}
\newcommand{\room}{721}
\newcommand{\others}{$\square$}
\newcommand*{\unit}[1]{\mathop{}\!\mathrm{#1}}
\newcommand*{\pderiv}[2]{\frac{\partial #1}{\partial {#2}}}
\newcommand*{\pderivh}[3]{\frac{\partial^{#1} #2}{\partial {#3^{#1}}}}
%% 如果是调课、补课, 改为: $\square$\hspace{-1em}$\surd$
%% 否则, 请用: $\square$
%%%%%%%%%%%%%%%%%%%%%%%%%%%

\begin{document}

%若需在页眉部分加入内容, 可以在这里输入
% \pagestyle{fancy}
% \lhead{\kaishu 测试}
% \chead{}
% \rhead{}

\begin{center}
    \LARGE \bf 《\, 基\, 础\, 物\, 理\, 实\, 验\, 》\, 实\, 验\, 报\, 告
\end{center}

\begin{center}
    \noindent \emph{实验名称}\underline{\makebox[25em][c]{\experiName}}
    \emph{指导教师}\underline{\makebox[8em][c]{\supervisor}}\\
    \emph{姓名}\underline{\makebox[6em][c]{\name}}%%如果名字比较长, 可以修改box的长度"5em"
    \emph{学号}\underline{\makebox[10em][c]{\studentNum}}
    \emph{分班分组及座号} \underline{\makebox[5em][c]{\class \ -\ \group \ -\ \seat }\emph{号}} (\emph{例}:\, 1\,-\,04\,-\,5\emph{号})\\
    \emph{实验日期} \underline{\makebox[3em][c]{\dateYear}}\emph{年}
    \underline{\makebox[2em][c]{\dateMonth}}\emph{月}
    \underline{\makebox[2em][c]{\dateDay}}\emph{日}
    \emph{实验地点}\underline{{\makebox[4em][c]\room}}
    \emph{调课/补课} \underline{\makebox[3em][c]{\others\ 是}}
    \emph{成绩评定} \underline{\hspace{5em}}
    {\noindent}
    \rule[8pt]{17cm}{0.2em}
\end{center}

\begin{center}
\LARGE{弦上驻波及介质中声速测量}
\end{center}

\section{实验目的}


\begin{enumerate}

    \item 观察在两端固定的弦线上形成的驻波现象, 了解弦线达到共振和形成稳定驻波的条件; 
	
	\item 测定弦线上横波的传播速度、确定弦线作受迫振动时共振频率和半波长个数$n$、弦线有效长度、张力及线密度之间的关系;
	
	\item 用对数作图、最小二乘法对共振频率与张力关系实验结果作线性拟合,并比较理论值和测量值,给出结论. 
    
    \item 利用驻波法和位相法测定空气和水中的超声波声速.
\end{enumerate}












\section{实验器材}

\subsection{弦上驻波实验}
本实验装置由弦音计、信号发生器和双踪示波器三部分组成. 另有测量工具, 如天平. 

\begin{enumerate}

    \item 弦音计由吉他弦、固定吉他弦的支架和基座、琴码、砝码支架、驱动线圈、
    探测线圈和砝码等组成。驱动线圈(通过信号发生器提供一定频率的功率信号产生交变磁力, 
    使得金属弦线振动)和探测线圈(将弦线的振动转换为电信号, 并接入示波器进行观察)是本装置的重要部分。
    下面是装置的示意图:
    
    \begin{figure}[htb]
        \begin{minipage}[t]{0.6\linewidth}
            \centering
            \includegraphics[height=3.5cm]{弦音计实验总装置图.jpg}
            \caption{弦音计实验总装置图}
        \end{minipage}
        \begin{minipage}[t]{0.39\linewidth}
            \centering
            \includegraphics[height=3.5cm]{弦线所受张力的示意图.jpg}
            \caption{弦线所受张力的示意图}
        \end{minipage}
    \end{figure}
    
    
    \item 信号发生器为低频功率信号发生器, 其输出信号的频率从$10\unit{Hz}$到$1\unit{kHz}$, 用于提供上述频率范围中具有一定功率的正弦信号, 以驱动线圈运动. 
	
    \item 双踪示波器用于观察信号源的波形并显示由探测线圈接收到的弦线振动的波形, 以便观察弦线的振动. 

\end{enumerate}

\subsection{介质中声速测量}

SW-2 型声速测量仪, 信号发生器, 示波器等(测量水中声速时需要略微改变装置)







% \begin{figure}[H]
%     \centering
%     \includegraphics[width=8cm]{图1.jpg}
%     \caption{}
% \end{figure}








\section{实验原理}

如果弦线的长度和波长之间满足某种关系,
使得当前进波和许多反射波都具有相同的相位时,
弦线上各点都作振幅各自恒定的简谐振动。
这时,弦线上有些点振动的振幅最大,称为波腹;
而另外有些点的振幅为零,称为波节,这便形成驻波现象。

相邻两波节 (或波腹) 的间隔距离$D$为波长$\lambda$的一半,
 称为半波长, 即$D=\frac{\lambda}{2}$. 由于弦线两端固定, 
 故而弦线两端均为波节, 那么弦线长度应为半波长的整数倍,
  记弦线长度为$L$, 则
\[
    L = nD = \frac{n\lambda}{2} \Longrightarrow \lambda = \frac{2L}{n} \qquad n=1,2,\,\cdots
\]

设振动频率为$f$, 则横波沿弦线的传播速度为$v=f\lambda$.
根据波动理论, 记拉紧的弦上张力为$T$,
弦线线密度为$\mu$, 波在传播方向 (与弦线平行) 的位置坐标为$x$, 
 振动位移为$y$, 那么沿弦线传播的横波应满足运动方程如下: 
\[
    \pderivh{2}yt = \frac{T}{\mu} \pderivh{2}yx  \qquad  \pderivh{2}yt = v^2\pderivh{2}yx
\]

注意,第二个式子是在我们令平面简谐波的方程为$\xi =A\cos\left[\omega \left(t+\frac{x}{v}\right)+\phi\right]$的基础上,并对$x,t$求二阶偏导得到的。

进而对比两式,可以得到波的传播速度满足:
\[
    v = \sqrt{\frac{T}{\mu}}
\]
对比其与$v = f\lambda$之间的差异, 
分析理论值与测量值间的区别. 

另外, 频率与张力、线密度间的关系为:
\[
    f = \frac{1}{\lambda} \sqrt{\frac T\mu}
\]
两边同时取对数, 则有(取标准单位制)
\begin{framed}
\begin{equation*} \label{eq:声速对数}
    \ln\lambda = \frac{1}{2}\ln T - \frac{1}{2}\ln\mu - \ln f
\end{equation*}
\end{framed}
这个式子是我们第一部分实验的处理和分析数据的关键。

当两个振幅和频率相同的相干波在同一直线上相向传播时,
其所叠加而成的波称为驻波。在弦线上出现许多静止点, 称为驻波的波节, 
相邻两波节间的距离为半个波长. 根据这个关系我们可以求解弦线上的波长。


\subsection{利用驻波法测声速}

将信号发生器输出的正弦电压信号经超声发射换能器电声转换为超声波发射出去,
 经由接受换能器声电转换为电压信号送入示波器. 
 若接收面与发生面严格平行, 入射波在接收面上垂直反射, 
 入射波、反射波相互干涉形成驻波, 
 此时两换能器之间距离恰好等于其声波半波长的整数倍,
可以从接收换能器端面声压的变化来判断超声波是否形成驻波. 

转动鼓轮,改变两个换能器间的距离, 
记录出现最大电压数值时标尺上的刻度, 
相邻两次最大值对应的刻度值之差即为半波长. 
频率$f$已知(最开始便确定), 经由上述方式测得波长$\lambda$, 
则可根据公式$v = \lambda f$可算出超声波的传播速度$v$. 

\subsection{利用相位法测声速}

将发射波和接收波同时输入示波器, 以X-Y模式显示, 
两波的频率相同, 相位不同. 
当接受点与发射点的距离变化恰等于波长的整数倍时, 
相位差为$2\pi$的整数倍, 等效为相位差为$0$. 
实验过程中, 通过改变发射器和接收器之间的距离, 
观察李萨如图形的变化进而观察相位变化, 从而可以得到波长。
根据公式$v = \lambda f$可求出波速$v$. 

当相位变化时, 部分李萨如图形如下: 

\begin{figure}[htbp]
    \centering
    \includegraphics[height=4cm]{不同的李萨如图形.jpg}
    \caption{频率相同、相位不同时的李萨如图形}
\end{figure}

关于声速的理论值,等到具体数据处理时再说明。



































\section{实验内容概要}
\begin{kaishu}
    注:这里撰写一些基本和通用的操作。关于我自己具体的实验操作以及遇到的情况、处理方法等在“实验结果与数据处理”部分穿插叙述。
\end{kaishu}

\bigskip
\noindent 【关于实验一】

(1)认识和调节仪器(此处吴老师已经演示得很详细了). 将信号发生器的一个端口和示波器的一个通道连接(并分出一支与激励线圈相连), 并将探测线圈连接到示波器的另一通道. 
    
(2)测定所用弦线的线密度. 用天平测定弦线 (选用与所用弦线直径相同、只取吉他弦中段约$70$-$80 \unit{cm}$的专用样品, 测量弦的线密度) 的质量$m$, 并测量弦线长$L$, 则线密度为: $\mu=\frac mL$
	
(3)观察弦线上的驻波. 固定弦上张力$T$与波的有效长度$L$, 调节信号发生器的输出频率, 观察在两端固定的弦线上形成的有$n\; (n=1,2,3,\cdots)$个波腹的稳定驻波. 
	
(4)测定弦线上横波的传播速度. 有两种方法:
 (i) 测得张力$T = \frac{1}{2}nmg$与线密度$\mu$, 
 根据$v=\sqrt{\frac T\mu}$测得横波传播速度(记录为理论值). 
 (ii) 测出共振频率$f$, 波的有效长度$L$, 根据$\lambda=\frac{2L}{n}$求得波长, 
 再利用$v=\lambda f$计算得到横波传播速度(记录为测量值). 比较两种方法得到的实验结果. 
	
(5)固定弦线线密度$\mu$与弦线有效长度$L$, 确定弦线作受迫振动时的共振频率$f$ (只取基频, 即$n=1$) 与弦线张力$T$之间的关系, 并记录数据. 
	
(6)固定弦线线密度$\mu$与弦线张力$T$, 确定弦线作受迫振动时的共振频率$f$ (只取基频, 即$n=1$) 与弦线有效长度$L$之间的关系, 并记录数据. 
	
(7)固定弦线张力$T$, 弦线线密度$\mu$与弦线张力$T$, 确定弦线作受迫振动时的共振频率$f$与半波长个数$n$之间的关系, 并记录数据. 
	
(8)固定弦线张力$T$、弦线有效长度$L$, 确定弦线作受迫振动时的共振频率$f$ (只取基频, 即$n=1$) 与弦线线密度$\mu$之间的关系, 并记录数据(此处建议共享数据). 

\bigskip

\noindent【关于实验二】

(1) 利用驻波法和位相法测超声波在空气中的波速(仪器连接与实验一很相似).

 (2) 利用驻波法或位相法测超声波在水中的波速. 

(3)利用逐差法处理实验数据(此外,我还利用了绘图法处理数据). 


















\section{实验结果与数据处理:第一部分}


\subsection{线密度测试}

首先是测量一些基本数据,我所在得桌号为10,在抽屉里取出样本弦进行质量(天平测量)、直径(螺旋测微器)、长度(钢尺)测量,并记录数据。
测量时我发现弦样本的粗细和仪器上的弦不太符合(相差最大时快达到0.100{mm}!),
经和吴老师反应并确定弦样品能接上后,我反复测量了弦样品和仪器所用弦的不同部位的直径,取了二者相近时的值作为记录。
结果见下表:
\begin{table}[H]
    \centering
    \caption{线密度测试:实验数据}
    \begin{tabular}{ccccc}
        \toprule
        弦号 & 质量(g) & 长度(mm) & 直径(mm) & 线密度$\rm Kg/m$ \\ 
        \midrule
        10 & 0.217 & 57 & 0.871 & 0.00381 \\ 
        \bottomrule
    \end{tabular}
\end{table}

值得注意的是,这里“线密度”的计算式是$m/L$,并不是$m/(\frac{d^2\pi}{4}L)$。
最开始我还认为数据表上的单位有误。


\subsection{波速的测量}
对于砝码质量,利用老师演示时就已调零的天平进行测量。
而关于张力的计算,采用经验公式$T=\frac{1}{2}\rm nmg$,其中n为悬挂点位置格数,$g$我取的是北京的重力加速度值$9.812\,\,{\rm m/s^2}$。
从而根据$v=\sqrt{T/\mu}$就可计算“理论值”。表中$f_1,f_2,f_3$分别表示一次、二次、三次谐波时对应的频率。

现在将两个琴码分别放在 150 mm 和 650 mm 处(则用于控制弦线的有效长度为 500 mm)。
将将钩码置于第 2(3、4) 个挂点位置,调节仪器左下角的旋钮直到其水平(观察气泡)。
随后将激振器与信号发生器输出端相连,并开启信号发生器(幅值调至最大20Vpp),在理论频率附近寻找形成
驻波的基频。

当频率选择正确时,弦会发出较大的声音(或者观察出明显振动,可利用示波器辅助观察),
弦线上会形成一个(后续要依次调出两个、三个)稳定的波腹,记下对应的频率。
实验中记录的数据见下表:
\begin{table}[H]
    \centering
    \caption{波速的测试:实验数据}
    \begin{tabular}{cccccccc}
        \toprule
        砝码位置 & $f_1$(Hz) & $f_2$(Hz)  & $f_3$(Hz)  & 波速($v=\lambda f$) & 张力(T) & 波速$v=\sqrt{{\rm T}/ \mu }$ & 相对误差$\eta$ \\ 
        \midrule
        2 & 37.78 & 75.68 & 113.83 & 37.85 & 4.979 & 36.15 & 4.49\% \\ 
        3 & 45.12 & 90.34 & 134.24 & 45.01 & 7.469 & 44.28 & 1.62\% \\ 
        4 & 54.49 & 108.96 & 160.12 & 54.11 & 9.958 & 51.12 & 5.53\% \\ 
        \bottomrule
    \end{tabular}
\end{table}

关于波速的计算,采用求均值的方法:
\[
  v=\frac{\lambda_1f_1+\lambda_2f_2+\lambda_3f_3}{3}=\frac{f_1+\frac{1}{2}f_2+\frac{1}{3}f_3}{3} \,\,{\rm m/s} 
\]

表2中最右侧我计算了相对误差,可以看到实验的测量还是比较精确的。
吴老师看后认为“经验公式”系数的$\frac{1}{2}$不太准确是主要误差来源(在【额外思考与讨论】部分我对其做了讨论)。
下两图是实验过程中的记录:


\begin{figure}[H]
    \centering
    \subfigure[波节记录]{\includegraphics[height=4.5cm]{1.jpg}}\hspace{0.5cm}
    \subfigure[示波器记录]{\includegraphics[height=4.5cm]{2.jpg}}
    \caption{波速的测量:实验照片}
\end{figure}

在实验中能观察到波节和弦的振动(二次谐波时),在拍摄中我尽量呈现出这一点。右图记录的数据是砝码位置为2时$f_3$的示波器数据。
这一实验较为顺利,初步体会了实验装置和测量原理,实验结果也较为精确(控制在$5\%$左右)。




\subsection{频率与有效长度的关系}
接下来,将利用控制变量法研究驻波的频率与其他因素的关系,其中驻波频率均取为基频$f_1$。
这一子实验探究频率与有效长度的关系,将砝码固定在第2个位置,移动琴码调节有效长度。
给出了五个有效长度,分别调节好并形成基频驻波。将对应的频率记录下来。
实验中测量到的数据如下:
\begin{table}[H]
    \centering
    \caption{频率与有效长度的关系:实验数据}
    \begin{tabular}{cccccc}
        \toprule
        L & 640\,\,mm & 480\,\,mm & 320 \,\,mm& 240\,\,mm & 160\,\,mm \\ 
        \midrule
        $f_1$ & 30.77 & 38.60 & 57.87 & 76.83 & 117.33 \\ 
        \bottomrule
    \end{tabular}
\end{table}
由此可以做出直线拟合图像(标准单位制之下,数值取自然对数后):
\begin{figure}[H]
    \centering
    \includegraphics[width=16cm]{图片1.png}
    \caption{频率与有效长度的关系:拟合图像}
\end{figure}
因为测量的是基频,故$\lambda=2L$,从而理论上存在关系式(取标准单位制):
\begin{align*}
    \ln f&=\frac{1}{2}\ln T-\frac{1}{2}\ln \mu-\ln\lambda\\
    &=-\ln L-\ln 2+\frac{1}{2}\ln T-\frac{1}{2}\ln \mu \\
    &=-\ln L-\ln 2+\frac{1}{2}\ln 4.979-\frac{1}{2}\ln 0.00381\\
    &= -\ln L+2.8945
\end{align*}
将上述拟合结果与理论值进行比较,可以得到下表:
\begin{table}[H]
    \centering
    \caption{频率与有效长度的关系:实验结果对比}
    \begin{tabular}{ccc}
        \toprule
        斜率 & 截距 \\ 
        \midrule
        测量值 & -1.0262 & 2.9635 \\ 
        理论值 & -1.0000 & 2.8945 \\ 
        相对误差 & 2.62\% & 2.38\% \\ 
        \bottomrule
    \end{tabular}
\end{table}
该数据与理论值在实验误差允许范围内已足够精确,直线拟合的$R^2=0.9982$,说明二者具有很强的相关性,从而本实验最终来看时成功的。

但值得一提的是,本次实验我反复测量了四次!
前两次都是在第三个数据点$L=320\,\,{\rm mm}$处再找不到基频谐振频率:
我最开始在预期范围内找,但后续一度到70、80Hz都没有现象,数据只能废除。
我反映情况后,吴老师指出可能是琴码不够高,没有卡住弦线导致的,
便让我用一张卫生纸垫着进行测量——但效果仍不明显,而且示波器上显示接收线圈的波形幅值变化很小。
第三次测量也作废,我只得借用邻近实验台的琴码进行实验,终于在第四次测量时得到了较预期中的数据。

实验数据处理时还出现一个小插曲——我没有取标准单位制,将$L(\rm mm)$直接用于数据处理,结果曲线斜率不变但截距变为了$9-10$之间——
原本准备写误差分析,讨论一下本实验的数据为何偏差这么大。
但突然发现并没有统一转化为标准单位制,并正确处理后,得到了比较精确的实验结果。这也反过来说明“统一标准”的重要性。

实验操作还是比较简单的,但我因为实验器材的缘故和自己检查得不细心,导致这个实验浪费了较多时间——这也给了我一些教训和经验。

















\subsection{频率与张力的关系}
现在开始测量频率与张力的关系,固定有效长度为$L=400\,\,{\rm cm}$,并通过改变砝码位置(每次放置都需要重新调平)
的方式改变张力,由此记录下基频频率。我的实验数据记录如下:
\begin{table}[H]
    \centering
    \caption{频率与张力的关系:实验数据}
    \begin{tabular}{cccccc}
        \toprule
        位置 & 1 & 2 & 3 & 4 & 5 \\ 
        \midrule
        T & 2.489 & 4.979 & 7.469 & 9.958 & 12.448 \\ 
        $f_1$ & 32.06 & 45.18 & 55.34 & 64.10 & 70.92 \\ 
        \bottomrule
    \end{tabular}
\end{table}
由此可以做出直线拟合图像与理论曲线(在标准单位制之下,数值取自然对数后):
\begin{figure}[H]
    \centering
    \includegraphics[width=16cm]{图片2.png}
    \caption{频率与张力的关系:拟合图像}
\end{figure}

理论曲线根据理论关系式做出:
\begin{align*}
    \ln f&=\frac{1}{2}\ln T-\frac{1}{2}\ln \mu-\ln\lambda\\
   &=\frac{1}{2}\ln T-\frac{1}{2}\ln 0.00381-\ln(2\times 0.400)\\
   &=\frac{1}{2}\ln T-3.0082
\end{align*}
将上述拟合结果与理论值进行比较,可以得到下表:
\begin{table}[!ht]
    \centering
    \caption{频率与张力的关系:实验结果对比}
    \begin{tabular}{ccc}
        \toprule
        ~&斜率 & 截距 \\ 
        \midrule
        测量值 & 0.4958 & 3.0159 \\ 
        理论值 & 0.5000 & 3.0082 \\ 
        相对误差 & 0.84\% & 0.26\% \\ 
        \bottomrule
    \end{tabular}
\end{table}

在上次实验的惨痛教训下,这次实验我细致检查了实验器材,
并尽可能地记录下最精确的谐振频率
(示波器显示接受线圈对应波形幅值最大、弦线明显振动时)。
本次实验数据的精度很高($R^2=0.9999$,两个相对误差都小于$1.00\%$!),
可能也是这次总的实验中我最满意的一个地方吧。

我在测量本次子实验数据时,已经有同学开始下一实验的“共享数据”甚至超声波声速的测量了。
但我尽量放慢速度,不受其他人进度干扰,
还请几位同学帮我检查数据的合理性和观察实验现象,
实验整体还是流畅的(总算不是像上一实验的反复测量四次了)。

虽然慢,但总算得到了很精确的结果——让我欣慰。




\newpage

\subsection{频率与线密度的的关系}
本次实验“共享数据”的形式,我记录到的数据如下(11号数据作图偏差较大,我便采用了另一组同学的7号数据),如下:
\begin{table}[H]
    \centering
    \caption{频率与线密度的关系:实验数据}
    \begin{tabular}{ccccccc}
        \toprule
        弦号 & 10 & 6 & 2 & *11 & 9 & 7 \\ 
        \midrule
        直径(mm) & 0.871 & 1.07 & 0.820 & *0.865 & 0.639 & 0.250 \\ 
        $\mu(\rm Kg/m)$ & 0.00381 & 0.00563 & 0.00354 & *0.00378 & 0.00203 & 0.00430 \\ 
        $f_1$ & 45.18 & 38.88 & 48.70 & *49.30 & 62.70 & 130.65 \\ 
        \bottomrule
    \end{tabular}
\end{table}

由此便可做出直线拟合图像以及理论曲线(还是一样地采用标准单位制,并对数值作取自然对数处理):
\begin{figure}[H]
    \centering
    \includegraphics[width=16cm]{图片3.png}
    \caption{频率与线密度的关系:拟合图像}
\end{figure}

理论上存在关系式:
\begin{align*}
    \ln f&=\frac{1}{2}\ln T-\frac{1}{2}\ln \mu-\ln\lambda\\
    &=-\frac{1}{2}\ln \mu+\frac{1}{2}\ln 4.979-\ln(2\times 0.400)\\
    &=-\frac{1}{2}\ln \mu+1.0258
\end{align*}

于是对比直线拟合数据以及理论值,可以得到下表:
\begin{table}[H]
    \centering
    \caption{频率与线密度的关系:实验结果对比}
    \begin{tabular}{ccc}
        \toprule
         ~&斜率 & 截距 \\ 
         \midrule
        测量值 & -0.4746 & 1.1932 \\ 
        理论值 & -0.5000 & 1.0258 \\ 
        相对误差 & 5.08\% & 16.32\% \\ 
        \bottomrule
    \end{tabular}
\end{table}

可以看到,直线的拟合效果很好$R^2=0.9989$,
但斜率和截距的相对误差较大(分别为$5.08\%$与$16.32\%$)。
因为是共享的数据,而且我是“二次转录”——借用了一位同学的该实验的记录表,
确认是本组数据后选取了几个数据进行记录,不太确定误差来源。

但我还是思考了一些可能的主要误差来源,列举如下:

·第一,不同同学测量的细心程度不一样,包括对实验器材的检查(弦线是否拉直,琴码是否足够卡住弦线等等)。
还有实验仪器也不统一,这也会造成一定的误差。

·第二,由于频率为基频时不会出现波节,所以大家基本是通过弦振动发出声
音响度最大以及振幅是否最大来判断频率是否达到基频。显然这个方法并不可靠,
而且由我的实验处理经验,些许的基频频率测定数据的变化便可引起相对误差个位数的变化!



针对第二条,我认为可以将测量基频改为“测量二次谐振频率”,这样或许能得到更精确的数据,希望以后的实验讲义能作一些更改。


















\section{实验结果与数据处理:第二部分}

这个实验需要重新连线:首先断开示波器、信号发生器等连线,将弦音计整理好放在一边,拿出超声波实验仪。
测量空气中超声波传播速度时,在两端口固定圆筒状超声换能器;其余连线与第一部分很类似,不再赘述。

使用驻波法测量时,将示波器调节为Y-T模式;而在位相法时调节为X-Y模式。
\subsection{测量空气中的超声波波速}

下表是我本次实验测量得到的实验数据(其中部分数据的详细运算在后面):
\begin{table}[H]
    \centering
    \caption{空气中的超声波波速测试:实验数据}
    \begin{tabular}{ccccc}
        \toprule
        \multicolumn{5}{l}{$f=\underline{\,40{\rm K}\,}({\rm Hz})\quad \text{室温t=}\underline{\,24.2\,} 
        ^\circ {\rm C}\quad V_{\text{理论值}}=\underline{\,345.82\,}{\rm m/s} $}\\
        \bottomrule
        \toprule
        i & 驻波法$L_i$(mm) & $\lambda_i\,\,(\rm mm)$ & 位相法$L_i$(mm) & $\lambda_i\,\,(\rm mm)$ \\ 
        \midrule
        1 & 49.895 & 8.803 & 42.375 & 8.68375 \\ 
        2 & 54.071 & 8.804 & 50.590 & 8.81875 \\ 
        3 & 58.660 & 8.780 & 59.591 & 8.74475 \\ 
        4 & 62.920 & 8.8605 & 68.330 & 8.7735 \\ 
        5 & 67.501 & 8.800 & 77.110 & 8.75275 \\ 
        6 & 71.679 & 
        \multirow{5}*{\makecell{逐差结果\\$\overline{\lambda}=8.8095\,\,{\rm mm}$}} & 85.865 &
         \multirow{5}*{\makecell{逐差结果\\$\overline{\lambda}=8.7547\,\,{\rm mm}$}} \\ 
        7 & 76.220 & ~ & 94.570 & ~ \\ 
        8 & 80.641 & ~ & 103.424 & ~ \\ 
        9 & 85.101 & ~ & 112.121 & ~ \\ 
        10 & 89.412 & ~ & 120.924 \\ 
        \bottomrule
        \toprule
        \multicolumn{5}{l}{测量结果:$V_{\text{实验值}}=\underline{\,352.38\,}\,\,(\rm m/s)$\qquad  $\mid$ \, \quad
        测量结果: $V_{\text{实验值}}=\underline{\,350.19\,}\,\,(\rm m/s)$}\\
        \bottomrule
    \end{tabular}
\end{table}

下图即为位相法实验过程中的实验图像(只是某个调节过程,实验中记录的是呈相同方向直线时对应的刻度值):
\begin{figure}[H]
    \centering
    \includegraphics[width=12cm]{5.jpg}
    \caption{位相法测量空气中超声波声速:实验记录}
\end{figure}

现在按照吴老师要求利用逐差法处理数据。对于驻波法,我采用的是记录相邻两个极大值之间的刻度值;对于位相法,我记录的是改变$2\pi$相位时的刻度值,故:
\[
  \lambda_{\text{驻波法}i}=\frac{\lambda_{i+4}-\lambda_i}{4}\times2  \qquad \lambda_{\text{位相法}i}=\frac{\lambda_{i+4}-\lambda_i}{4}
\]
最后“逐差结果”指的是将该列四个数据求平均,这便得到了表中对应的数据。

声波在空气中传播的理论速度为
\[
   v=v_0\sqrt{\frac{T}{T_0}
   }=v_0\sqrt{1+\frac{t\,\,(^\circ{\rm C})}{273.15}}=331.45\times\sqrt{1+\frac{24.2}{273.15}}\,\, {\rm m/s}=345.82\,\,{\rm m/s}
\]
而根据逐差法计算出的实验数据$\overline \lambda$,由$v=\overline \lambda f$,得:
\begin{align*}
    &v_{\text {实验值·驻波法}} = 352.38\,\,{\rm m/s}\qquad \eta_1\approx 1.90\%\\
&v_{\text {实验值·位相法}} = 350.19\,\,{\rm m/s}\qquad \eta_2\approx 1.26\%
\end{align*}


假如我们使用绘图法,用直线拟合上述两组数据,可以得到以下图像:

\begin{figure}[H]
    \centering
    \includegraphics[width=15cm]{图片4.png}
    \caption{测量空气中的超声波波速:绘图法}
\end{figure}

根据实验记录的操作,可知
\begin{align*}
    &v'_{\text {实验值·驻波法}} =2\times 4.40513\times 40\,\,{\rm m/s} =352.41\,\,{\rm m/s}&\eta'_1\approx 1.91\%\\
&v'_{\text {实验值·位相法}} =8.75332\times40\,\,{\rm m/s}= 350.13\,\,{\rm m/s}  &  \eta'_2\approx 1.24\%
\end{align*}

可以看到本实验的测量还是很精确的——相对误差都控制在$2.00\%$以内。下面列举一些我的思考:

·实验误差的来源主要是驻波法“极大值”和位相法“重合”的判断——这很考验实验者的经验判断能力,
而且由于回程差较大,只能调节鼓轮向一个方向。
其他的误差来源:例如,使用的材料并非是纯水,
声波接触到容器壁后会反弹,
读取刻度时的误差,旋转转轮时传动装置不同步带来的误差等。

·为了解决上述问题,我采取了“预实验”的方式——一是判断变化的大致范围与间距;
二,还发现了当间距较小时接收器得到的幅值较大,甚至可能会超出屏幕,这促使我选取合适的量程。

·在正式测量时,每当快当记录点时,我都很小心地调节——
由于示波器波形有延迟甚至鼓轮不动时
波形都在缓慢变化(在这个实验还不太明显,
下一实验中我深刻地体会到了这个问题带来的实验不便之处)——尽可能地提高实验数据的精度。



















\subsection{测量水中的超声波波速}

最后我采用位相法测量水中的超声波波速。

将连线断开,取下换能器圆柱,将装有适量水的水槽放在超声波实验仪正前方。
然后换上水中使用的探头,组装装置:将水下探测器的两根支杆分别插入超声波实验仪
的固定槽中(保证水槽中水面完全没过两探头),并使水中两探测器面接近平行,与水槽的侧壁垂直并留有一定间隔,高度
大致相同,最后紧固所有螺丝。我测量得到的实验数据记录如下:
\begin{table}[H]
    \centering
    \caption{水中的超声波波速测试:实验数据}
    \begin{tabular}{ccc}
        \toprule
        \multicolumn{3}{l}{$\text{方法:}\underline{\text {\,位相法\,}}\quad f=\underline{\,1.7{\rm M}\,}({\rm Hz})
        \quad \text{室温t=}\underline{\,24.2\,} ^\circ {\rm C}$}\\
       \bottomrule
        \toprule
        i & 刻度值$L_i$(mm) & $\lambda_i\,\,(\rm mm)$ \\ 
        \midrule
        1 & 12.861 & 0.9195 \\ 
        2 & 13.337 & 0.836 \\ 
        3 & 13.787 & 0.869 \\ 
        4 & 14.224 & 0.8025 \\ 
        5 & 14.700 & 0.790 \\ 
        6 & 15.009 &\multirow{5}*{\makecell{逐差结果\\$\overline{\lambda}=0.8434\,\,{\rm mm}$}} \\ 
        7 & 15.525 & \\ 
        8 & 15.829 &  \\ 
        9 & 16.280 &  \\ 
        10 & 16.742 &\\ 
        \bottomrule
        \toprule
        \multicolumn{3}{l}{测量结果:$V_{\text{实验值}}=\underline{\,1433.78\,}\,\,(\rm m/s)$}\\
        \bottomrule
    \end{tabular}
\end{table}

下图即是实验过程中的记录(对应最开始调整好频率时):
\begin{figure}[H]
    \centering
    \includegraphics[width=15cm]{7.jpg}
    \caption{测量水中的超声波波速:位相法}
\end{figure}


现根据逐差法处理数据——由于我测量时记录的是相位每改变$\pi $时对应的刻度值,从而:
\[
    \lambda_{\text{位相法}i}=\frac{\lambda_{i+4}-\lambda_i}{4}\times 2
\]

由《温度与水中声速对照表》(1个标准大气压下)可知,$t=24.2\,^\circ{\rm C}$时,水中声速值为$v_{\text 水 }=1494.40\,\,{\rm m/s}$


根据逐差法得到的数据,可知:
\[
  v_{\text{测量值}}=1433.78\,\,{\rm m/s}  \qquad \eta\approx 4.03\%
\]

而如果绘图,用直线拟合的结果去计算,则有
\begin{figure}[H]
    \centering
    \includegraphics[width=16cm]{图片5.png}
    \caption{测量水中的超声波波速:绘图法}
\end{figure}
从而可以得到:
\[
  v_{\text{测量值}}=2\times 0.42395\times170\,\,{\rm m/s}=1441.14\,\,{\rm m/s}  \qquad \eta\approx 3.56\%
\]

本次实验的精度还是比较高的——控制在$4\%$左右,如果考虑到鼓轮细微变化都会较显著影响到波形。
而且实验的理论值也不一定是真的“理论值”,吴老师说在$1400-1600\,\,{\rm m/s}$范
围内都是正常的,由此看本实验最终是成功的。

但实验过程却异常艰难——第一次测量,就在图10所示的位置(10号台)进行,由于细微调整就会引起显著变化(有时甚至不动,延迟很严重,或者一动改变几个周期!)
又不能回转调节确定错过的值,我的试验记录数值间隔甚至不统一。当时看着同学们都已经快做完,便怀揣着侥幸心理,
认为总之记录的是改变$\pi$的整数倍时的刻度变化,
处理数据时跨度较大的两个数据差除以合适的整数倍(当然那时需要参考一下其他同学的间隔值),
就能得到改变$\pi$时的间隔值,也能得到实验结果。便提交了实验数据记录表。
但吴老师异常细心严谨,一眼就发现了我的数据记录得有误,让我很是羞愧。

重新开始实验——我尝试了我的原有位置对应的实验仪器,还是很不稳定。于是更换到已经做完实验的一位同学的实验台,
但发现还是无法记录到数据——甚至无法开始记录——观察我的原始数据记录表就能发现我尝试了十余次的记录。
最后在不懈努力下,总算能完整记录一组实验数据,结束了本次非常艰难的实验——五点几分结束,
是进行了九次物理实验的过程中,第一次在晚于四点半后结束实验。

\section{额外思考与讨论}

\subsection{对张力经验公式的修正}
在弦上驻波部分中,张力与砝码重力的关系公式$T=\frac12\rm nmg$只是经验公式,因此系数不一定是
准确的,而且根据波速的测量实验结果与理论计算的偏差,有必要对其进行修正并进行一些讨论。

仍然假设二者呈简单的线性关系,即$T=k\rm nmg$,其中 k 为待修正的系数。
从而根据$\overline v=\sqrt{k{\rm nmg}/\mu}$,可得:
\[
    k=\frac{\overline v^2\mu}{\rm nmg}
\]
代入实验测量的三个砝码位置时的测量值,得到:
\[
  k_2=\frac{37.85^2\times 0.00381}{2\times 507.46\times 10^{-3}\times 9.812} =0.548\quad
  k_3=0.516\quad k_4=0.560 
\]
取平均值作为结果,并得到与“标准值”$\frac12$的相对偏差:
\[
   \overline k=\frac{k_1+k_2+k_3}{3}= 0.541\qquad \eta=8.26\%
\]
可以看到还是有一定偏差的。
但综合考虑“频率和张力关系”实验结果又很精确,
修正过程所得结果仍不具有太大的可信度,因为数据收集的过程本身就有一定的误差
(也不排除由于计算式子的原因,把两个实验的“误差”给集中到波速实验了)。

\subsection{处理数据时的“统一单位制”}

在“弦上驻波”实验讲义中,将波长、张力、密度、频率四个物理量取对数再研究其关系。
这样其实不太严谨:物理量是有量纲的,
取对数操作对“单位”有什么影响尚且不清楚。

故而,应该是统一为标准单位制后再对“数值”取自然对数。比如在本实验中,取
\[
   f\sim {\rm Hz}\quad T\sim{\rm N}\quad L,\lambda\sim{\rm m}\quad \mu\sim{\rm Kg/m}
\]
实验数据的处理中我也确实因为这个原因有所困惑,
结果发现是因为$L$没有取标准单位制发现拟合直线截距有较大偏差
(可参见“频率与有效长度的关系”实验)。

另外,考虑到这点,前面呈现的拟合图像的横纵坐标都是没有单位的,认为只是在处理数值本身。




\section{思考题}

\begin{kaishu}
    注:本部分回答实验讲义上的思考题(只有第一部分实验)。
\end{kaishu}


\begin{enumerate}
    \item 调节振动源上的振动频率和振幅大小后对弦线振动会产生什么影响?
    
    \begin{kaishu}
        调节振动频率会相应地改变弦线振动的频率。如果振动源的振动频率是弦线基频的
整数倍,则弦线发生共振,形成稳定驻波,此时实验现象较为明显。若不是共振时,那么弦线上会形成行波,即波峰与波谷的位
置会向一个方向移动,此时不便于观察。
    
          调节振动源的振幅自然会以同样的形式影响弦线振动的振幅,在未形成驻波时不太
明显,形成驻波时肉眼可见振幅变大,会使弦线的振动更加明显,振动声音也更大,便
于观察判断驻波的形成。
    \end{kaishu}
   
    \item 如何来确定弦线上的波节点位置?
    
    \begin{kaishu}
        一是理论计算。可以根据振动的源频率和有效部分长度知道有几个波节。
        
        二是实验观察。可以通过肉眼观察或拍照,一直不振动的点即为波节点。如果肉眼不易观察,
        那么可以移动探测器的位置,在移动时观察到振幅为零或最小的位置,即为波节点。
        当然,如果一直观察不到,也有可能是实验操作的不得当处使得波节消失,需要检查一下实验器材。

    \end{kaishu}
  
    \item 在弦线上出现驻波的条件是什么? 
    在实验中为什么要把弦线的振动调到驻波现在最稳定、最显著的状态?
    
    \begin{kaishu}
         条件是弦长是半波长的整数倍(激振器需要处于合适的位置),因为此时频率对应的才是某个共振频率。 
 
         调到最稳定、最显著的状态,可以观察到最明显的驻波现象,使得误差减小,
计算结果更加精确。在实验中,在一个频率范围内都可能会看到驻波现象,只是强度、
稳定性有所差别,所以将驻波调到最稳定、显著,有助于确定形成“真正”驻波的频率值,
从而减小数据的误差。

    \end{kaishu}
   
    \item 在弹奏弦线乐器时, 发出声音的音调与弦线的长度、粗细、松紧程度有什么关系? 为什么?
    
    \begin{kaishu}
        首先,音调由频率决定,频率越高,音调越高。 从而弦长越短、固定越紧, 
发出声音的音调也就越高。这是由公式(或者从我们的实验结果)得到的。

但关于直径对其的影响,需要思考一下(或者参考下一问题的回答):首先弦直径会影响横截面积从而影响线密度的大小。
另外,较粗的弦会振动阻力更大,而且有可能粗细不均——这些都会影响驻波的产生。
从而,对于实际需求比如演奏而言,只需要弦粗细均匀即可,不必过于纠结其粗细。

    \end{kaishu}
   
    \item 若样品弦线与装置上的弦线直径略有差别, 请判断是否需要修正, 如何进行?
    
    \begin{kaishu}
        需要修正。线密度有所不同,从而会对实验造成影响。
        利用千分尺测量样品弦与装置上的弦的直径,得出它们的比,
        如两种弦材质相同,密度相同,则利用其横截面积的比就可以得到装置上的线密度。
    \end{kaishu}
  
    \item 对于某一共振频率, 
    增大或减少频率的调节过程中, 
    振幅最大的频率位置往往不同, 
    如何解释这一现象?
    
    \begin{kaishu}
         可能原因有很多:
         
    第一, 驻波建立好之后, 
    若继续调整向远离基频的地方, 
    振动还未稳定 (驻波被破坏之前) 频率就被调整了, 
    导致在某一频率处观察到的的振幅与实际振幅不相同. 

    第二, 驻波形成后发出声音, 这是一种能量散失的形式, 
    声音导致弦线受到更多的阻力, 使得最初的微分方程不再成立, 
    振幅最大的频率位置随之变化.

    第三,由于弦的振动是受迫振动,所以改变频率时,需要一段的弛豫时间,
    如果调节的速度过快,可能会导致振幅最大的位置频率不同。
    
    第四,也应当考虑仪器本身的误差。或者可能存在一些不可预见的误差。
    \end{kaishu}
   
\end{enumerate}



\section{感想总结}


“弦上驻波及介质中声速测量”实验可以算是我整个物理实验中最“不顺利”的实验——在已经完成的前八个实验中,我都是组内最先完成、实验操作完整度很好而且数据精度较高的同学,还经常帮助其他同学完成一些困难的仪器连线等。
但这个实验我最终到5:07分才完成数据记录和实验台整理。

在实验过程中也有诸多不顺——比如最开始发现弦线直径变化较大,第三个小实验整整测量了四组才得到完整数据(竟然是因为琴码没有卡住弦线!)
,最后测量水中超声波波速实验的数据读取也很艰难(本来就需要测量很小的位移,示波器显示还有较大延迟,我更换了新实验台也是如此)——
这些都让本次实验看起来“简单”,但对我来说测量过程却很漫长。

不过,也有一些让我欣慰的地方:

1、在测量频率与有效长度的关系之后的“频率与张力的关系”实验,得到的实验数据符合理论值得很好。
可能也和我耐下性子,尽可能仔细地寻找精确数据有关。

2、吴老师耐心严谨的教导。我每次有疑问时,吴老师都能及时回应。
甚至最后一次实验数据,我原本想反正记录的是$\pi $变化的整数倍对于的刻度值,处理数据时乘除合适整数倍就可以。
但吴老师还是指出了这样的不合理性,并让我重做——在我完成实验后还鼓励我“实验数据是可以的,只是你要耐心一点”,我对此印象深刻。
这也算一次对我心智的锻炼磨砺吧。
\smallskip

回顾整个实验:

·【实验一】本次实验观察了在两端固定的弦线上形成的驻波现象,用两种方法测定了弦线上横波的传播速度并进行了误差分析和比较,对弦线达到共振和形成稳定驻波的条件有了更深的了解。
本次实验确定了弦线作受迫振动时共振频率与半波长个数n、弦线有效长度、张力及弦密度之间的理论关系:
$\ln\lambda = \frac{1}{2}\ln T - \frac{1}{2}\ln\mu - \ln f$并利用实验数据对其进行了验证。

·【实验二】本次实验利用相位法和驻波法两种方法测量了空气中的声速,
使用逐差法、绘图法处理实验数据,发现相位法的精确度较高。
本次实验利用相位法测量了水中的声速,
并和纯水中的理论值相比较,计算了相对误差。实验整体精度较高。
\smallskip

最后,我还有以下的思考与感悟:

·“打铁还需自身硬”。在正式做实验之前,必须先做好预习工作。比如通过认真绘制预习实验报告中的实验原理和流程操作,对实验有整体的把握。

·体悟理论与实验相结合的物理魅力:物理首先是一门实验学科,对它的学习如果只是从书本上看,从课堂上
听,而没有动手亲自“学”的话,终究对其的理解和把握是有限的。在此之前,我对驻波原理、声速理论值计算等
已经有所了解,但真正像这样动手测量并观察实验现象的过程中,我相信对它们的认识又加深了一层。

总的来看,本次实验是成功、富有意义、印象深刻的。驻波原理简单而深刻,而声速测量实验的设计也是巧妙的。
我有幸能通过本次实验对这两部分有所认识和感悟。




















\vspace*{5cm}
——————————

附:

1、预习实验报告(手写pdf版导入)

2、原始实验数据记录表(包含老师签名)(扫描为pdf版导入)

\newpage

\includepdf[pages={1-2}]{11_草稿.pdf}
\includepdf[pages={1-2}]{扫描全能王 2023-12-13 00.26.pdf}



\end{document}