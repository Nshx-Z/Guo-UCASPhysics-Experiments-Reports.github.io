%%%%%%%%%%%%%%%%%%%%%%%%%%%%%%%%%%%%%%%
%                                     %
%   %    %   %  %   %%%%%    %  %  %  %
%  %%   %%   %  %   %       %%  %  %  %
%   %    %   %%%%%  %%%%%    %  %%%%% %
%   %    %      %       %    %     %  %
%  %%%  %%%     %   %%%%%   %%%    %  %
%                                     %
%%%%%%%%%%%%%%%%%%%%%%%%%%%%%%%%%%%%%%%

%本实验报告由本人林诚皓和吉骏雄一起完成, 旨在方便LATEX原教旨主义者写实验报告, 避免Word文档因插入过多图造成卡顿. 

\documentclass[11pt]{article}

\usepackage[a4paper]{geometry}
\geometry{left=2.0cm,right=2.0cm,top=2.5cm,bottom=2.5cm}

\usepackage{ctex}
\usepackage{amsmath,amsfonts,graphicx,subfigure,amssymb,bm,amsthm}
\usepackage{algorithm,algorithmicx}
\usepackage[noend]{algpseudocode}
\usepackage{fancyhdr}
\usepackage{mathrsfs}
\usepackage{mathtools}
\usepackage[framemethod=TikZ]{mdframed}
\usepackage{fontspec}
\usepackage{adjustbox}
\usepackage{breqn}
\usepackage{fontsize}
\usepackage{tikz,xcolor}

\setmainfont{Palatino Linotype}
\setCJKmainfont{SimHei}
\setCJKsansfont{Songti}
\setCJKmonofont{SimSun}
\punctstyle{kaiming}

\renewcommand{\emph}[1]{\begin{kaishu}#1\end{kaishu}}

%改这里可以修改实验报告表头的信息
\newcommand{\experiName}{气轨上弹簧振子的简谐振动及瞬时速度的测定}
\newcommand{\supervisor}{姚楚豪}
\newcommand{\name}{李果}
\newcommand{\studentNum}{2022K8009906028}
\newcommand{\class}{01}
\newcommand{\group}{09}
\newcommand{\seat}{01}
\newcommand{\dateYear}{2023}
\newcommand{\dateMonth}{10}
\newcommand{\dateDay}{16}
\newcommand{\room}{716}
\newcommand{\others}{$\square$}
%% 如果是调课、补课, 改为: $\square$\hspace{-1em}$\surd$
%% 否则, 请用: $\square$
%%%%%%%%%%%%%%%%%%%%%%%%%%%

\begin{document}

%若需在页眉部分加入内容, 可以在这里输入
% \pagestyle{fancy}
% \lhead{\kaishu 测试}
% \chead{}
% \rhead{}

\begin{center}
    \LARGE \bf 《\, 基\, 础\, 物\, 理\, 实\, 验\, 》\, 实\, 验\, 报\, 告
\end{center}

%不要忘了预习报告这个前缀可能还需要修改!

\begin{center}
    \noindent \emph{实验名称}\underline{\makebox[25em][c]{\experiName}}
    \emph{指导教师}\underline{\makebox[8em][c]{\supervisor}}\\
    \emph{姓名}\underline{\makebox[6em][c]{\name}}%%如果名字比较长, 可以修改box的长度"5em"
    \emph{学号}\underline{\makebox[10em][c]{\studentNum}}
    \emph{分班分组及座号} \underline{\makebox[5em][c]{\class \ -\ \group \ -\ \seat }\emph{号}} (\emph{例}:\, 1\,-\,04\,-\,5\emph{号})\\
    \emph{实验日期} \underline{\makebox[3em][c]{\dateYear}}\emph{年}
    \underline{\makebox[2em][c]{\dateMonth}}\emph{月}
    \underline{\makebox[2em][c]{\dateDay}}\emph{日}
    \emph{实验地点}\underline{{\makebox[4em][c]\room}}
    \emph{调课/补课} \underline{\makebox[3em][c]{\others\ 是}}
    \emph{成绩评定} \underline{\hspace{5em}}
    {\noindent}
    \rule[8pt]{17cm}{0.2em}
\end{center}

\begin{center}
\Large{气轨上弹簧振子的简谐振动及瞬时速度的测定}
\end{center}

\section{实验目的}

1. 学习气垫导轨和数字毫秒计的使用。

2. 观察简谐振动现象和运动学特征,测定简谐振动的周期。

3. 验证机械能守恒定律。

4. 用极限法测定瞬时速度,深入了解平均速度和瞬时速度的关系。

\section{实验器材}

气垫导轨、滑块、附加砝码、弹簧、U 型挡光片、平板挡光片、数字毫秒计、天平等。


\begin{figure}[htbp]
    \centering
    \includegraphics[width=12cm]{qddj1.png}
    \caption{气垫导轨示意图及各部件}
\end{figure}

\section{实验原理}

\subsection{弹簧振子的简谐运动}

弹簧振子的简谐振动原理是简洁的,其具体方程为:\begin{displaymath}-k_1(x+x_0)-[-k_1(x-x_0)]=m\ddot x\end{displaymath}令$k=2k_1$即可得到标准的微分方程:\begin{displaymath}kx=m\ddot x\end{displaymath}这个方程不仅在数学中的微分方程和物理理论上,还是在实际应用中都很重要。解这个方程得:\begin{displaymath}x=A\sin (\omega_0t+\varphi_0)\end{displaymath}其中$A$是振幅,$\varphi_0$是初相位,$\displaystyle \omega_0=\sqrt{\frac{k}{m}}$是固有频率。

考虑到$T$与$\omega_0$的关系,可以得到:\begin{displaymath}T^2=4\pi^2\frac{m_1+m_0}{k}\end{displaymath}就可以得到$T^2$与$m$之间的线性关系,从而方便作图。

\subsection{简谐振动的运动学特征描述}

对上面的简谐振动的微分方程解考虑,对其求一阶导即可得到$x-v$有关的关系,解得:\begin{displaymath}v^2=\omega_0^2(A^2-x^2)\end{displaymath}就可以得到$x$与$v$的周期变化规律及其相位关系。

\subsection{简谐振动的机械能}

简谐振动的动能\begin{displaymath}E_k=\frac{1}{2}mv^2\end{displaymath}以及势能\begin{displaymath}E_p=\frac{1}{2}kx^2\end{displaymath}根据这两者的表达式可以求出其机械能总和为:\begin{displaymath}E=\frac{1}{2}kA^2\end{displaymath}因而在前面的小实验中测量其位移与速度,可以计算其动能与势能,进而验证其转换关系与机械能守恒定律。

\subsection{瞬时速度的测定}

在实验中我们无法准确地按照导数亦或微分的定义,真正做到时间趋于0,从而完成对瞬时速度的测定。实际上,我们采用近似的方法:

我们利用挡光片与光电门可以测得$\Delta s,\Delta t$,减少前者使其趋于0,做出相关曲线并使其延伸到$\Delta t=0$处,即可得到瞬时速度的近似值。这种测量方法运用了极限和外推的思想。

\section{实验内容}

\subsection{学会光电门测速和测周期的使用方法}

按照姚老师演示的步骤,先连接气泵电源并打开,然后连接光电计数器的电源,打开开关,将光电门连接线连接到光电计数器上。
按光电计数器的功能按钮可以调节至不同功能,比如测量周期、瞬时速度等。

值得注意的是,实验开始时,应该先开气源,后放滑块;实验结束时,应先取滑块,再关气源。

\subsection{实验仪器的调试}

先对气垫导轨进行调平,具体步骤为:

粗调:调整气垫导轨直到滑块位于中间时保持静止或轻微的左右来回移动。

精调:装好光电门和计数器,小滑块上安装条形挡光片,将小滑块从左到右或从右到左以一定初速度推动,记录两光电门速度的读数,当两速度误差小于百分之0.5时视为调平。

\subsection{测量弹簧振子的振动周期并考察振动周期和振幅的关系}

滑块的振幅分别取10.0,20.0,30.0,40.0cm时进行测量,按照理论,如果弹簧振子做简谐振动,那么在不同振幅下周期应该保持不变,通过分析和讨论实验结果验证这一结论是否正确。

要注意的是,滑块要轻拿、轻放;挂弹簧时,要特别小心,一定要手扶住滑块。

\subsection{研究振动周期和振子质量之间的关系}

在滑块上加骑码(铁片)。对一个确定的振幅,譬如40cm,的情况下,每增加一个骑码测量一组T(注意骑码不能加太多,以阻尼不明显为限)进而做出$T^2-m$曲线,并用最小二乘法直线拟合,求出$k,m_0$的值。

\subsection{研究速度和位移的关系}

在滑块上装上 U 型挡光片,可测量速度。具体来说,我们可测量通过挡光片的时候,滑块的平均速度速度,此时可近似看成瞬时速度。进而做出$v^2-x^2$的曲线,同样进行线性拟合。

有一点被反复提及:测量周期时需使用条形挡光片挡光,测量速度时 U 形挡光片挡光。

\subsection{研究振动系统的机械能是否守恒}

取40cm的固定振幅,测出在不同$x$处的速度,由此计算经过每一个$x$处的动能和势能,求出各处的机械能并进行比较,得出结论。

\subsection{研究平均速度与瞬时速度的关系}

利用外推法求出瞬时速度。在气轨下面只有一个螺丝的那一端,小心将气轨抬起来,把垫块放到这个螺丝的下面。改变$\Delta s$并由此求出平均速度$\bar{v}$,做出图像并由此进行线性外推,从而得到瞬时速度。

需要说明的是,测定瞬时速度时,在滑块左右两侧分别装上 U 形挡光块,并且把挡光片放在滑块前部,以滑块的前缘即是挡光片的前沿为好,另外一块挡光片放在滑块对面后沿,以保证前后配重均衡。且每次一定静止释放滑块,当滑块滑过光电门后,一定及时用手扶住滑块,防止滑块撞到后部弹簧上。

注意:禁止用手把滑块按压在导轨上的方式来给滑块减速。

\subsection{重复实验}

通过增减垫块数量的方式改变倾斜角度,并通过改变设置距离的方式,重复上述实验,从而得到更有一般性意义的结果。

此外,需要说明的是,更换、安装或者调节挡光片在滑块上的位置时,或者放骑码时,都必须把滑块从导轨上取下来,待调节或者安装好后再放上去。

\section{数据记录与处理}

\subsection{实验仪器的调试}

气垫导轨的调平是本节实验的最大的难点。但在认真倾听姚老师的讲解后,我很迅速并优秀地完成了这一步骤。

调平的操作需要放置1cm挡光片后,使用两个光电门(这也是本实验唯一一次同时做到使用两个光电门的过程),放置到相距较远的位置,测出其速度,求差并除以较小的那个速度值,即所需的“误差/百分比”。

下面是我的调制结果,见表1:

\begin{table}[!ht]
    \centering
    \begin{tabular}{|l|l|l|}
    \hline
        $v_1$(cm/s) & $v_2$(cm/s) & 误差(百分比) \\ \hline
        28.80 & 28.70 & 0.34 \\ \hline
        25.27& 25.34 & 0.28 \\ \hline
        30.16 & 30.24 & 0.27 \\ \hline
    \end{tabular}
    \caption{调试实验仪器}
\end{table}

事实上表1列举出来的结果包含了从左到右,再从右到左的速度。三次测量的误差小于百分之0.5,调平过程相当成功。

\subsection{测量弹簧振子的振动周期并考察振动周期和振幅的关系}

下面是实验开始的第一部分。本实验的全程,其实就是包含着两个部分的——第一部分是需要弹簧的简谐振动,而第二部分是瞬时速度的测量。

这一小部分考察弹簧振子的振动周期并探究其与振幅的关系(其实并没有什么关系,我们需要实验验证这一点)。
我们先换成测量周期的条形挡光片,连接上弹簧,把光电门(仅留一个)放在平衡位置附近(但是不要重合)。
数字毫秒计调到周期模式,以十个周期为一组,选择取数操作,即可开始记录。

相关的实验原始数据记录如下表2:

\begin{table}[htbp]
    \centering
    \begin{tabular}{|c|c|c|c|c|}
    \hline
    周期T \textbackslash 振幅A & 10cm    & 20cm    & 30cm    & 40cm    \\ \hline
    T1(ms)                 & 1587.24 & 1585.60 & 1586.61 & 1586.07 \\ \hline
    T2(ms)                 & 1588.27 & 1585.24 & 1586.57 & 1585.94 \\ \hline
    T3(ms)                 & 1588.59 & 1585.86 & 1586.51 & 1586.12 \\ \hline
    T4(ms)                 & 1588.13 & 1585.95 & 1586.49 & 1586.08 \\ \hline
    T5(ms)                 & 1588.59 & 1586.28 & 1586.61 & 1586.23 \\ \hline
    T(ms)                  & 1588.16 & 1585.79 & 1586.54 & 1586.09 \\ \hline
    \end{tabular} 
    \caption{测量弹簧振子的振动周期并考察振动周期和振幅的关系}
\end{table}
   
    
对比表2的最后一行,即周期平均值一栏,(结合下面根据实验数据绘制的图2)可以看出,其误差值是相当小的。
在误差允许的范围内,我们验证了弹簧振子的运动周期与振幅无关,这是与理论相符合的。这部分实验成功完成。

\begin{figure}[htbp]
        \centering
        \includegraphics[width=14cm]{图片1.png}
        \caption{T与A的关系图表}
\end{figure}

\newpage
\subsection{研究振动周期和振子质量之间的关系}

第二部分部分研究振动周期和振子质量之间的关系。由于我们需要控制变量,故而这一部分改变质量,就保持振幅A不变,恒定为40cm。
首先,需要用天平称量一些物品的质量,见表3。


\begin{table}[htbp]
    \centering
    \begin{tabular}{|c|c|}
    \hline
    物体              & 质量/g   \\ \hline
    滑块              & 215.36 \\ \hline
    条形挡光片           & 2.59   \\ \hline
    U型挡光片(10cm/1cm) & 11.8   \\ \hline
    U型挡光片(5cm/3cm)  & 11.48  \\ \hline
\end{tabular} 
\caption{部分物体的质量}
\end{table}

实际实验中,我采取了安装好滑块上的挡光片和大、小骑码后,在测量这一小实验数据之前进行质量测量。表3的数据,可以作为后续实验的参考。
值得说明的是,在下面的表4中,我们将1状态设定为只有滑块与遮光片的质量,在此基础上(这两者是必不可少的),2为增加了小骑码,3为增加的大骑码,4为增加了一大、一小两个骑码,而5为增加了两个大骑码的状态,且大骑码1出现在状态3、4、5之中,而大骑码2仅在状态5出现。这样的设定是合理的。
此外,在使用电子天平的时候,需要在重启后,等待示数为0,方可进行操作。

在此基础上进行测量,得到表4:

\begin{table}[htbp]
    \centering
    \begin{tabular}{|c|c|c|c|c|c|}
    \hline
    周期T/质量m(g) & 217.95  & 230.42  & 242.84  & 255.25  & 267.65  \\ \hline
    T1(ms)     & 1586.40 & 1630.43 & 1673.02 & 1713.21 & 1753.11 \\ \hline
    T2(ms)     & 1586.52 & 1630.43 & 1673.12 & 1713.48 & 1753.06 \\ \hline
    T3(ms)     & 1586.66 & 1630.46 & 1672.98 & 1713.66 & 1753.03 \\ \hline
    T4(ms)     & 1586.59 & 1630.78 & 1673.03 & 1713.67 & 1753.08 \\ \hline
    T5(ms)     & 1586.55 & 1630.10 & 1673.21 & 1713.74 & 1353.32 \\ \hline
    T6(ms)     & 1586.68 & 1630.44 & 1673.24 & 1713.75 & 1753.41 \\ \hline
    T7(ms)     & 1586.73 & 1630.56 & 1673.13 & 1713.63 & 1753.35 \\ \hline
    T8(ms)     & 1586.77 & 1630.65 & 1672.97 & 1713.63 & 1753.29 \\ \hline
    T9(ms)     & 1586.87 & 1630.59 & 1672.97 & 1713.73 & 1753.44 \\ \hline
    T10(ms)    & 1586.99 & 1630.54 & 1673.17 & 1713.88 & 1753.22 \\ \hline
    T(ms)      & 1586.68 & 1630.50 & 1673.08 & 1713.64 & 1753.23 \\ \hline
    \end{tabular}   
\caption{研究振动周期和振子质量之间的关系}
\end{table}

事实上,必须说明的是,表2中40cm即最后一列和表4第二列,即情况一时对应的数据相同,我们看到数据符合得很不错($1586.09\leftrightarrow 1586.68$)。

这一部分需要我们进行线性拟合,对数据进行更深层次的处理。我们有公式:\begin{displaymath}T^2=4\pi^2 \frac{m+m_0}{k}\end{displaymath}
请注意,我们此时将质量作为一个整体来看待,结合表3测得的数据,这样考虑也是完全合理的。因而,我们做表5:

\begin{table}[htbp]
    \centering
    \begin{tabular}{|c|c|}
    \hline
    m(g)   & T\textasciicircum{}2/(ms)\textasciicircum{}2 \\ \hline
    217.95 & 2517553.422                                  \\ \hline
    237.82 & 2658530.250                                   \\ \hline
    242.84 & 2799196.686                                  \\ \hline
    255.25 & 2936562.050                                   \\ \hline
    267.65 & 3073815.433                                  \\ \hline
    \end{tabular}
    \caption{数据处理:周期与质量}
    \end{table}

并由此做出图3:

\begin{figure}[htbp]
    \centering
    \includegraphics[width=14cm]{图片2.png}
    \caption{周期-质量图象}
\end{figure}

可以看到,线性拟合的结果还是不错的($R=0.9584$)。由此我们通过上面的公式进行计算,得出:
\[k\approx 3.5114\quad kg/s^2\]

由截距可得弹簧的质量:
\[m_0\approx 6.6311\,\, g\]

\subsection{改变振幅测劲度系数}

这一部分本来应该是第六个小实验做的,但考虑到其是作为第三个小实验的补充、乃至在验证机械能守恒定律的小实验中起到修正作用而存在的,将实验结果和相关数据处理提前至这。
当然其理论上也有其自身的价值。

此外,从这一部分开始,就不是周期的测量了,因而也要相应地,先更换挡光片。

测量得到表6:

\begin{table}[!ht]
    \centering
    \begin{tabular}{|l|l|l|l|l|l|}
    \hline
        - & 10cm & 15cm & 20cm & 25cm & 30cm \\ \hline
        $v_{max1}$(cm/s) & 37.39 & 58.17 & 77.58 & 96.90 & 116.55 \\ \hline
        $v_{max2}$(cm/s) & 37.91 & 57.64 & 78.00 & 96.25 & 116.28 \\ \hline
        $v_{max3}$(cm/s) & 37.51 & 58.04 & 78.25 & 96.43 & 116.69 \\ \hline
        $v_{max}$(cm/s) & 37.60 & 57.95 & 77.94 & 96.53 & 116.51 \\ \hline
    \end{tabular}
    \caption{最大速度与振幅的关系}
\end{table}

根据\begin{displaymath}V_{max}^2=\frac{k}{m+m_0}A^2\end{displaymath}可以再由此做出表7:

\begin{table}[htbp]
    \centering
    \begin{tabular}{|l|l|}
    \hline
    $A^2(m^2)$ & $v^2_{max}(m^2/s^2)$ \\ \hline
    0.0100                                       & 0.14137600                                                                 \\ \hline
    0.0225                                     & 0.33582025                                                               \\ \hline
    0.0400                                      & 0.60746436                                                               \\ \hline
    0.0625                                     & 0.93180409                                                               \\ \hline
    0.0900                                       & 1.35745801                                                               \\ \hline
    \end{tabular}
    \caption{数据处理:最大速度和振幅}
\end{table}

根据作图结果画出图4:

\begin{figure}[htbp]
    \centering
    \includegraphics[width=14cm]{图片3.png}
    \caption{利用振幅与最大速度的关系求解K}
\end{figure}

根据该实验中$m=217.95\,\, g$,$m_0=6.6311\,\, g$进而求出:
\[k\approx 3.3983\quad kg/s^2\]

我们看到线性拟合的效果特别好($R=0.9998$),说明我们本次实验的成功。

事实上,这两次的实验测出的$k$值相差并不大,我们取其二者平均为准。一般来说,选择多组情况平均作为最终测量结果的方法,也是比较科学的。

\subsection{研究速度和位移的关系}

根据姚老师的提醒,在这个小实验中,我采用了只使用一个光电门的处理方式,这不会让我手忙脚乱,增加出错的可能性。

我们需要固定振幅为A=40cm,测量得到表8:

\begin{table}[!ht]
    \centering
    \begin{tabular}{|l|l|l|l|l|l|}
    \hline
        - & 10cm & 15cm & 20cm & 25cm & 30cm \\ \hline
        $v_1$(cm/s) & 148.81 & 142.65 & 132.45 & 117.92 & 99.11 \\ \hline
        $v_2$(cm/s) & 149.09 & 143.68 & 133.51 & 118.20 & 99.80 \\ \hline
        $v_3$(cm/s) & 147.93 & 143.06 & 132.45 & 117.73 & 100.10 \\ \hline
        $v$(cm/s) & 148.61 & 143.13 & 132.80 & 117.95 & 99.67 \\ \hline
    \end{tabular}
    \caption{研究速度和位移的关系}
\end{table}

这一部分也是需要直线拟合的。我们进行数据处理,具体见表9:

\begin{table}[!ht]
    \centering
    \begin{tabular}{|l|l|}
    \hline
        $x^2$ & $v^2$ \\ \hline
        0.01 & 2.20849321 \\ \hline
        0.0225 & 2.04861969 \\ \hline
        0.04 & 1.763584 \\ \hline
        0.0625 & 1.39122025 \\ \hline
        0.09 & 0.99341089 \\ \hline
    \end{tabular}
    \caption{数据处理:速度与位移}
\end{table}

根据表9绘制图5:

\begin{figure}[htbp]
    \centering
    \includegraphics[width=14cm]{图片4.png}
    \caption{速度与位移的关系}
\end{figure}

可以看到,其线性相关性还是特别好($R=0.9989$),拟合出来的结果也很近似为一条直线。
由于\[v^2=\omega_0^2(A^2-x^2)\]

结合公式$y=-15.452x+2.3764$,可以看出其斜率与截距出来,进而可以求出在此数据下的$\omega_0\approx 3.9309$,
进而可以求得$\displaystyle T=\frac{2\pi}{\omega_0}\approx 1598.09\,\,ms$。与之前测得的结果$T_{40cm}=1586.09\,\, ms$在误差允许范围内符合。


\subsection{研究振动系统的机械能是否守恒}

这一部分,其实并不需要额外再去“测量”等做原始实验,仅需根据实验原理,进行数据处理即可。

根据前面几节(尤其是上一节)得到的数据,我们可以进行处理,采用我们在上面的小实验中得到的平均值$k\approx 3.4548 \,\, kg/s^2$,以及质量$m=217.95 \,\,g$,$m_0=6.6311\,\, g$,进而得到表10:

\begin{table}[!ht]
    \centering
    \begin{tabular}{|l|l|l|l|l|l|}
    \hline
        - & 10cm & 15cm & 20cm & 25cm & 30cm \\ \hline
        v(cm/s) & 148.61 & 143.13 & 132.80 & 117.95 & 99.67 \\ \hline
        E\_k(J) & 0.24780 & 0.23004 & 0.19803 & 0.15622 & 0.11155\\ \hline
        E\_p(J) & 0.01727 & 0.03887 & 0.06910 & 0.10796 & 0.15547 \\ \hline
        E(J) & 0.26527 & 0.26891 & 0.26713 & 0.26418 & 0.26702 \\ \hline
    \end{tabular}
    \caption{研究振动系统的机械能是否守恒}
\end{table}
辅以图表观察(见图6):
\begin{figure}[htbp]
        \centering
        \includegraphics[width=14cm]{图片6.png}
        \caption{动能、势能和总能量的关系图表}
    \end{figure}
我们首先根据理论上的计算,得到理论上的能量值应该为:$\displaystyle E=\frac{1}{2} kA^2\approx 0.276284\,\, J$。这与我们刚刚测得的结果相比,在误差允许的范围内,还是不大的,因而,我们的实验比较成功,机械能守恒定律基本上得以验证。

\newpage
\subsection{研究平均速度与瞬时速度的关系}

下面就是本次物理实验的第二部分。这一部分不需要弹簧,仅需要单独的滑块即可完成实验。

具体操作,大致来说,是先放垫片。直接记录时间,速度后续可自行处理。
此外,需要注意的是,此处的AP距离,指的是光电门与挡光片外侧的距离,且在释放滑块时,需要尽量使其初速度为0,否则将会造成干扰。
对于数字毫秒计,我们仍切换到$S^2$并使用毫秒(因为测量的不是速度),并保持初距离不变。

其实,三次不同的测量的实验原理都大同小异,我们已经全部介绍完毕。在第一次,我们设定AP为50cm,并添加了一块的高度。具体数据见表11:

\begin{table}[!ht]
    \centering
    \begin{tabular}{|l|l|l|l|l|l|l|}
    \hline
        挡光片宽度(cm) & $\Delta t_1$(ms) & $\Delta t_2$(ms) & $\Delta t_3$(ms) & $\Delta t_4$(ms) & $\Delta t_5$(ms) & $\Delta t$(ms) \\ \hline
        1(cm)     & 27.63  & 27.53  & 27.58    & 27.8     & 27.81  & 27.67  \\ \hline
        3(cm)     & 81.28   & 81.71  & 81.33    & 81.67    & 81.40  & 81.48   \\ \hline
        5(cm)     & 135.23  & 135.17 & 135.60   & 135.71   & 135.51  & 135.44  \\ \hline
        10(cm)    & 262.68  & 262.41 & 264.48   & 263.61   & 263.21  & 263.28  \\ \hline
    \end{tabular}
    \caption{测量瞬时速度的数据表-第一次}
\end{table}

同理,我们改变条件,再加上一块垫片以改变导轨倾斜角度,仍设定AP为50cm,得到的数据分别见表12:

\begin{table}[!ht]
    \centering
    \begin{tabular}{|l|l|l|l|l|l|l|}
    \hline
        挡光片宽度(cm) & $\Delta t_1$(ms) & $\Delta t_2$(ms) & $\Delta t_3$(ms) & $\Delta t_4$(ms) & $\Delta t_5$(ms) & $\Delta t$(ms) \\ \hline
        1(cm)     & 20.36  & 20.35  & 20.30  & 2038   & 20.31  & 20.34  \\ \hline
        3(cm)     & 60.03  & 60.21  & 60.23  & 60.24  & 60.14  & 60.17  \\ \hline
        5(cm)     & 99.82  & 100.05 & 99.88  & 100.10 & 99.97  & 99.96  \\ \hline
        10(cm)    & 195.26 & 195.17 & 195.59 & 195.74 & 195.29 & 195.40 \\ \hline
    \end{tabular}
    \caption{测量瞬时速度的数据表-第二次}
\end{table}

保持倾斜角度不变,改变AP距离为60cm,再重复一次实验,具体数据见表13:

\begin{table}[!ht]
    \centering
    \begin{tabular}{|l|l|l|l|l|l|l|}
    \hline
        挡光片宽度(cm) & $\Delta t_1$(ms) & $\Delta t_2$(ms) & $\Delta t_3$(ms) & $\Delta t_4$(ms) & $\Delta t_5$(ms) & $\Delta t$(ms) \\ \hline
        1(cm)     & 18.54 & 18.56 & 18.60  & 18.54 & 18.57  & 18.56  \\ \hline
        3(cm)     & 54.86 & 55.09 & 54.85  & 54.95 & 54.96  & 54.94  \\ \hline
        5(cm)     & 91.49 & 91.44 & 91.14  & 91.17 & 91.41  & 91.33  \\ \hline
        10(cm)    & 179.1 & 179.6 & 179.28 & 179.6 & 179.18 & 179.35 \\ \hline
    \end{tabular}
    \caption{测量瞬时速度的数据表-第三次}
\end{table}

进而我们可以分别求出其平均速度,根据\begin{displaymath}\bar{v}=\frac{\Delta s}{\Delta t}=v_0+\frac{a}{2}\Delta t\end{displaymath}
这两个关系式,我们就可以求出$\bar{v},\Delta t$的值,进而可以线性拟合,做出直线,
进而可以采用外推的思想得到我们想考虑的瞬时速度。

\newpage
从第一次开始,绘制表14:

\begin{table}[!ht]
    \centering
    \begin{tabular}{|l|l|}
    \hline
        $\Delta t(s)$ & $\bar{v}(m/s)$ \\ \hline
        0.02767 & 0.361402241 \\ \hline
        0.08148 & 0.368188513 \\ \hline
        0.13544 & 0.369167159 \\ \hline
        0.26328 & 0.379823762 \\ \hline
    \end{tabular}
    \caption{数据处理-第一次}
\end{table}

从而得到图7:

\begin{figure}[htbp]
    \centering
    \includegraphics[width=10cm]{图片7.png}
    \caption{第一次}
\end{figure}

可以求出$v_{0}^1\approx 0.3602\,\, m/s$.

同理进行第二次,绘制表15与图8:

\begin{table}[!ht]
    \centering
    \begin{tabular}{|l|l|}
    \hline
        $\Delta t(s)$ & $\bar{v}(m/s)$ \\ \hline
        0.02034 & 0.491642085 \\ \hline
        0.06017 & 0.498587336 \\ \hline
        0.09996 & 0.50020008  \\ \hline
        0.19540  & 0.511770727 \\ \hline
    \end{tabular}
    \caption{数据处理-第二次}
\end{table}

\begin{figure}[htbp]
    \centering
    \includegraphics[width=10cm]{图片8.png}
    \caption{第二次}
\end{figure}

可以求出$v_0^2\approx 0.4802\,\, m/s$.

\newpage
同理,第三次的话,绘制表16,图9:

\begin{table}[!ht]
    \centering
    \begin{tabular}{|l|l|}
    \hline
        $\Delta t (s)$ & $\bar{v}(m/s)$ \\ \hline
        0.01856 & 0.538793103 \\ \hline
        0.05494 & 0.546050237 \\ \hline
        0.09133 & 0.547465236 \\ \hline
        0.17935 & 0.557568999 \\ \hline
    \end{tabular}
    \caption{数据处理-第三次}
\end{table}

\begin{figure}[htbp]
    \centering
    \includegraphics[width=10cm]{图片9.png}
    \caption{第三次}
\end{figure}

可以求出$v_0^3\approx 0.5380\,\, m/s$.

\newpage
\section{思考题}

\begin{enumerate}
    \item 思考题1:  \textbf{仔细观察,可以发现滑块的振幅是不断减小的,那么为什么还可以认为滑块是做简谐振动?实验中应如何尽量保证滑块做简谐振动?}

振幅不断减小是由于存在阻尼,而我们之所以仍认为滑块作简谐振动,是因为气垫导轨摩擦较小,对实验结果的影响不是很大,事实上实际运动是欠阻尼简谐振动。在不考虑耗散的情况下,可近似认为是简谐运动。

实验中,我们使用气垫导轨以消除滑动摩擦;保证气垫导轨的水平;先打开气垫导轨再放置滑块;用光电门和数字毫秒计测量周期和速度,这些都可以尽量保证其做简谐振动。

    \item 思考题2:  \textbf{试说明弹簧的等效质量的物理意义,如不考虑弹簧的等效质量,则对实验结果有什么影响?}

因为实际上,弹簧本身的质量不可忽略,而质量的存在会导致弹簧内产生驻波,因而,严格来说,此时运动已经不满足简谐振动方程。为使简谐振动的原理尽可能被满足,亦即减小误差,我们考虑将这部分造成的影响用“等效质量”的概念代替:如果忽略高阶小量的话,那么就可以认为振动时仍保持质量线性分布,并且弹簧质量远小于滑块质量,由此大致上便解决了问题,更方便分析、计算。

如不考虑弹簧的等效质量,则可能会出现能量不守恒等情况,总机械能偏小,造成误差。

    \item 思考题3:  \textbf{测量周期时,光电门是否必须在平衡位置上?如不在平衡位置会产生什么不同的效果?}

理论上并不需要,这对我们的测量结果并没有影响。

不过,在实际操作的过程中,由于存在能量耗散,导致振幅减小,如果不在平衡位置将会不便测量,造成较大的误差。此外,不在平衡位置导致每次测量的时候不处于周期的同一位置,这也势必会很不方便测量。

    \item 思考题4:  \textbf{气垫导轨如果不水平,是否能进行该实验?}


理论上来说,如果只是为了观察简谐运动的性质,那么其实是不需要调平的,而且甚至于说,在忽略弹簧重力的情况下,机械能守恒的验证也并不会受到影响。只不过此时的公式会更加繁杂,需要考虑修正,这是不够便利的。

    \item 思考题5:  \textbf{使用平板形挡光片和两个光电门,如何测量滑块通过倾斜气轨上某一点的瞬时速度?}

目标还是测量“瞬时”速度,因而实验思路大致上其实是不变的,仍然是通过改变距离,进而改变$\Delta s$,并测得其相应的$\Delta t$,通过线性拟合得到类似的结论。

    \item 思考题6:  \textbf{气垫导轨如果不水平,对瞬时速度的测定有什么影响?}

并不影响。事实上,测定瞬时速度的实验,还会要求我们的气垫导轨水平。

    \item 思考题7:  \textbf{每次测量滑块和 U 型挡光片总质量不同是否对瞬时速度测定有影响?}

理论上应该是没有的,因为加速度与具体的质量值有关,进而,速度值也与具体的质量值无关。不过,不排除总质量不同导致表面积不同、进而导致阻力不同等情况造成的影响。

\end{enumerate}

\newpage
\section{感想与收获}

气垫导轨实验的实验原理本身比较简单——但这并不能说明我们可以掉以轻心——实验操作上需要仔细和耐心。
姚楚豪老师最开始详尽细致地向我们分析实验原理中的“小秘密”,以及如何在实验过程中利用这些原理辅助实验,接着实际操作实验仪器,每一步都耐心提醒注意事项——比如测量不同量时需要更换挡光片,调平时需要仔细耐心。
在这过程中,我积极地回答姚老师的问题,并和同学们分享自己的看法。

实验进行过程中,我有疑问的地方主动向老师提问,老师也耐心地给予了回复。这使得我迅速而精准地完成了逐项实验,成为最先递交实验原始数据记录表的同学。并且,根据后续对数据的处理分析结果,可以看到,本次实验是成功的。

但“实验”并没有结束,需要对较多数据进行处理分析和拟合,是这个实验的又一大特点。在撰写本份实验报告的过程中,我熟悉了\LaTeX、Excel等工具的基本用法,并且更进一步地熟悉了
利用Excel进行多种数据地处理汇总,并且绘制相应的图像,描绘相应的趋势线等等,对我后续物理实验的进行,或者以后的发展都是有利的。
物理归根到底是一门实验科学,姚老师在分析实验原理的过程中也提醒我们“实践出真知”、“纸上得来终觉浅,绝知此事要躬行”的道理。
或许弹簧振子的各个公式、关系我们在高中就已经很熟悉,但是不实际来做一次实验,又怎么会对其产生直接而深刻的体悟呢?这或许就是物理实验课程对我的意义——在理解物理实验原理的基础上,对实际的、真实的、非模型化的事物规律进行探索和思考。
在这一过程中,我更加熟悉了一些计算机软件的使用,这也是另一方面的收获吧。

再回顾整个实验,我在熟悉原理的基础上,对遇见的实验问题进行检查和分析,不理解的立即向老师询问,这或许是我本次实验成功的关键。后续的数据处理也让我增进了分析误差、绘制图表、归纳汇总等多方面的能力。

————————

(附,原始数据记录表,包含姚楚豪老师的教师签字,见下页)

\newpage

\begin{figure}[htbp]
    \centering
    \includegraphics[width=16cm]{IMG_1.jpg}
    \caption{第一页}
\end{figure}

\begin{figure}[htbp]
    \centering
    \includegraphics[width=16cm]{IMG_2.jpg}
    \caption{第二页}
\end{figure}

\begin{figure}[htbp]
    \centering
    \includegraphics[width=16cm]{IMG_3.jpg}
    \caption{第三页}
\end{figure}

\end{document}