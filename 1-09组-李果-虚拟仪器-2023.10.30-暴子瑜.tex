%%%%%%%%%%%%%%%%%%%%%%%%%%%%%%%%%%%%%%%
%                                     %
%   %    %   %  %   %%%%%    %  %  %  %
%  %%   %%   %  %   %       %%  %  %  %
%   %    %   %%%%%  %%%%%    %  %%%%% %
%   %    %      %       %    %     %  %
%  %%%  %%%     %   %%%%%   %%%    %  %
%                                     %
%%%%%%%%%%%%%%%%%%%%%%%%%%%%%%%%%%%%%%%

%本实验报告由本人林诚皓和吉骏雄一起完成, 旨在方便LATEX原教旨主义者写实验报告, 避免Word文档因插入过多图造成卡顿. 
\documentclass[11pt]{article}

\usepackage[a4paper]{geometry}
\geometry{left=2.0cm,right=2.0cm,top=2.5cm,bottom=2.5cm}

\usepackage{ctex}
\usepackage{amsmath,amsfonts,graphicx,subfigure,amssymb,bm,amsthm}
\usepackage{algorithm,algorithmicx}
\usepackage[noend]{algpseudocode}
\usepackage{fancyhdr}
\usepackage{mathrsfs}
\usepackage{mathtools}
\usepackage[framemethod=TikZ]{mdframed}
\usepackage{fontspec}
\usepackage{adjustbox}
\usepackage{breqn}
\usepackage{fontsize}
\usepackage{tikz,xcolor}
\usepackage{bigstrut,multirow,rotating}
\usepackage{booktabs}
\usepackage{multirow} 
\setmainfont{Palatino Linotype}
\setCJKmainfont{SimHei}
\setCJKsansfont{Songti}
\setCJKmonofont{SimSun}
\punctstyle{kaiming}

\renewcommand{\emph}[1]{\begin{kaishu}#1\end{kaishu}}

\newcommand{\unit}[1]{\,\text{#1}}
\newcommand{\cunit}[1]{\,#1}

%改这里可以修改实验报告表头的信息
\newcommand{\experiName}{虚拟仪器在物理实验中的应用}
\newcommand{\supervisor}{暴子瑜}
\newcommand{\name}{李果}
\newcommand{\studentNum}{2022K8009906028}
\newcommand{\class}{01}
\newcommand{\group}{09}
\newcommand{\seat}{8}
\newcommand{\dateYear}{2023}
\newcommand{\dateMonth}{10}
\newcommand{\dateDay}{30}
\newcommand{\room}{702}
\newcommand{\others}{$\square$}
%% 如果是调课、补课, 改为: $\square$\hspace{-1em}$\surd$
%% 否则, 请用: $\square$
%%%%%%%%%%%%%%%%%%%%%%%%%%%

\begin{document}

%若需在页眉部分加入内容, 可以在这里输入
% \pagestyle{fancy}
% \lhead{\kaishu 测试}
% \chead{}
% \rhead{}

\begin{center}
    \LARGE \bf 《\, 基\, 础\, 物\, 理\, 实\, 验\, 》\, 实\, 验\, 报\, 告
\end{center}

\begin{center}
    \noindent \emph{实验名称}\underline{\makebox[25em][c]{\experiName}}
    \emph{指导教师}\underline{\makebox[8em][c]{\supervisor}}\\
    \emph{姓名}\underline{\makebox[6em][c]{\name}}%%如果名字比较长, 可以修改box的长度"5em"
    \emph{学号}\underline{\makebox[10em][c]{\studentNum}}
    \emph{分班分组及座号} \underline{\makebox[5em][c]{\class \ -\ \group \ -\ \seat }\emph{号}} (\emph{例}:\, 1\,-\,04\,-\,5\emph{号})\\
    \emph{实验日期} \underline{\makebox[3em][c]{\dateYear}}\emph{年}
    \underline{\makebox[2em][c]{\dateMonth}}\emph{月}
    \underline{\makebox[2em][c]{\dateDay}}\emph{日}
    \emph{实验地点}\underline{{\makebox[4em][c]\room}}
    \emph{调课/补课} \underline{\makebox[3em][c]{\others\ 是}}
    \emph{成绩评定} \underline{\hspace{5em}}
    {\noindent}
    \rule[8pt]{17cm}{0.2em}
\end{center}

\begin{center}
\LARGE{虚拟仪器在物理实验中的应用}
\end{center}

\section{实验目的}

1. 了解虚拟仪器的概念.

2. 了解图形化编程语言 LabVIEW,学习简单的 LabVIEW 编程.

3. 完成伏安法测电阻的虚拟仪器设计.


\section{实验器材}

清华同方超越 E500-1002 型计算机一台(含操作系统),
LabVIEW2014,NIELVIS II+,导线若干,元件盒一个(包括100$\Omega$标准电阻一个,待测电阻1k$\Omega$和51$\Omega$各一个,稳压二极管一个)
,热电偶等元件.

\section{实验原理}

\subsection{虚拟仪器的硬件}
本实验使用的硬件平台是个人电脑(PC 机), 美国国家仪器公司(National Instruments)的教学实验室虚
拟仪器套件(Educational Laboratory Virtual Instrumentation Suite)II+(缩写为 NI ELVISII+)和自带
的原型板.

另外,我们所使用的NIELVIS II+及原型板类似于预科实验中见到的电路板(面包板),
但是其功能更多,可以充当电压表、电源等元件,而板上功能变化却不是很大,
极其方便使用.对于预科实验已经初步掌握电路板的同学而言,
只需要学习如何使用其新功能,进行电源和电压表的操作即可.

\subsection{虚拟仪器的软件}

本实验使用的用于虚拟仪器系统设计的软件开发平台是LabVIEW(laboratory virtual instrument engineering workbench).
它将计算机数据分析和显示能力与仪器驱动程序整合在一起,为针对仪器的编程提供了很大的便利.
而且,LabVIEW是一种图形化编程语言,编程过程也就是设计流程图.因而其对于毫无经验的初学者而言,也是相当友好.

用LabVIEW开发平台编制的虚拟仪器程序简称为VI.VI包括三个部分:前面板(front panel)、程序框图(Block diagram)和图标/连线板.

前面板用于设置输入数值和显示输出量,相当于真实仪表的前面板.
前面板上的图标,分为输入类和显示类,具体可以是开关、旋钮、按钮、图形、图表等表现形式.
程序框图相当于仪器的内部功能结构,其中的端口用来和前面板的输入对象和显示对象传递数据,
节点用来实现函数和功能子程序调用,图框用来实现结构化程序控制命令,连线则代表程序执行过程中的数据流.

我们使用两者相配合,相当于“造”出了一个新仪器.

\begin{figure}[htb]
    \centering
    \subfigure[硬件平台:NI\,\,ELVIS\,II+实物图]{\includegraphics[height=4cm]{图1.png}}\hspace{0.5cm}
    \subfigure[软件平台: Lab VIEW 软件]{\includegraphics[height=4cm]{图2.jpg}}
    \caption{本实验的硬件与软件平台}
\end{figure}

\subsection{创建一个温度测量程序}

一、创建一个模拟温度测量程序:假设有一个传感器, 其输出电压和温度成正比, 用它编写一个模拟温度测量的程序. 
假设当温度为 $80$华氏度时, 传感器输出电压为 $0.8 V$, 那么我们可以编写程序, 根据电压计算温度, 并且给出摄氏度和华氏度两种显示.

二、使用真实热电偶元件创立一个温度测量程序:假设传感器的输出电压和温度成正比,则可以利用程序根据电压计算温度.

\subsection{创建一个电压输出和采集的程序}
本实验通过编写输出/输入通道, 两个停止按钮, 可以手动调整的输出电压与测量到的电压, 连接电路, 
使用 While 循环使得程序每 $100 {ms}$ 输出/测量一次电压, 
并且在电路板上连接好两根导线, 得到的结果就可以随时返回前面板.

\subsection{利用虚拟仪器测量伏安特性}

我们利用一个模拟输出通道为整个测量电路供电,利用两个模拟输入通道分别测量总电压和标准电阻上的电压;
利用测量得到的电压数值和标准电阻数值就可以得到电路中的电流以及待测电阻上的电压.简单来说,其原理如图2所示:

% \begin{figure}[htbp]
%     \centering
%     \includegraphics[width=8cm]{图3.jpg}
%     \caption{}
% \end{figure}

\begin{figure}[htb]
    \centering
    \subfigure[利用虚拟仪器测量伏安特性原理图]{\includegraphics[width=8cm]{图3.jpg}}\hspace{0.5cm}
    \subfigure[实验三:用虚拟仪器测量伏安特性实际电路图]{\includegraphics[width=8cm]{qq_pic_merged_1698676578044.jpg}}
    \caption{原理图与实际电路图}
\end{figure}
在程序控制下,电路电压由0开始逐渐增加,电压每改变一次,
测得一组电压电流值,最后得到一个数组,经过线性拟合后就可以得到待测电阻值.
这就是测量伏安特性的基本原理了.

\section{实验内容}
\subsection{初步熟悉LabVIEW的开发环境}
打开 LabVIEW 2014; 在"文件"菜单中选择"新建V1", 我们可以看到前面板和程序框图. 
直接使用快捷键 Ctrl + T 并排两个窗口, 方便编程. 或者点击窗口,可以实现二者之间的切换.
在前面板的"查看"菜单中打开"控件选板"和"工具选板", 可以从"工具选板"中选择"自动选择"工具, 方便操作. 
可以在"控件选板"中新建"温度计"并将其显示出来. 

在程序框图窗口中, 
在"查看"菜单中打开"函数选板" 来显示函数选板, 
利用"函数选板"新建"加法", 并且尝试为之连线.

在下列实验开始之前, 
打开面板上的两个开关, 
即 ELVIS 电源 (在仪器后面) 和原型板电源 (在仪器上面的右上方).
并学习选择并放置控件、点击右键查看快捷菜单,学习使用标签工具、定位工具、连线工具,熟悉各种快捷键.

\subsection{创建一个温度测量程序}
\subsubsection{创建一个模拟温度测量程序}
(1)创建前面板

依次放入温度计(控件 - 数值- 温度计)、
垂直滑动杆开关(控件 - 布尔 - 垂直滑动杆开关,并改名为温度值单位,显示开关状态,继续使用标签工具使其显示“摄氏”与“华氏”)、
数值显示控件(控件 - 数值 - 数值显示控件,并改名为温度值)、
数值输入控件(控件 - 数值 - 数值输入控件,并改名为采集的电压)

(2)创建程序框图

利用程序框图在函数选板中放入乘法函数、减法函数、除法函数(函数 - 数值)、
选择函数(函数 - 比较,并且设置根据温标选择开关的值输出华氏温度或者摄氏温度数值)、
移动位置并利用连线工具连接起来,并在需要的地方创建数值常量.可以再整理一下图标位置和连线.

值得注意的是,这里由于选择函数的特性,"温度计"和"温度值"位置不能调换.

\begin{figure}[htb]
    \centering
    \subfigure[前面板]{\includegraphics[height=7cm]{温度 前面板.JPG}}\hspace{0.5cm}
    \subfigure[程序框图]{\includegraphics[height=7cm]{温度 程序.JPG}}
    \caption{实验一: 创建一个模拟温度测量程序}
\end{figure}

\newpage
(3)运行程序

运行VI程序,点击连续运行按钮,使程序运行于连续运行模式.改变“采集的电压”控件输入值(比如在0.5~2.0之间的任意值)和温度值单位,
观察程序运行情况,并解释程序每部分的功能.停止运行,并保存即可.
\begin{figure}[htbp]
    \centering
    \includegraphics[width=5cm]{温度 运行.JPG}
    \caption{实验一:模拟温度测量程序运行}
\end{figure}

\subsubsection{使用真实热电偶元件创立一个温度测量程序}
本实验也可以使用温度变送器,我们可以通过测量热电偶的电压,并利用Labview 内置的转换功能,实现温标的转化和温度的测量

(1)同前,创建前面板 

(2)创建程序框图
依次放入加法函数、乘法函数、选择函数(同前).
放入DAQ助手(采集信号 - 模拟输入 - 温度 - 热电偶),热电偶(K 型),
设备类型选择“ NI Elvis II+ ”,通道选择ai0.
同前面一样,利用连线工具连接,并在需要的地方创建数值常量,最后整理一下图标位置即完成了本次程序的创建.

我的程序截图及运行结果如下图:
\begin{figure}[htb]
    \centering
    \subfigure[前面板]{\includegraphics[height=6cm]{热电偶 前.JPG}}\hspace{0.5cm}
    \subfigure[程序框图]{\includegraphics[height=6cm]{热电偶 程序.JPG}}
    \caption{实验一: 使用真实热电偶元件创立一个温度测量程序}
\end{figure}


(3)利用香蕉转针型线连接电路即可

(4)运行程序
点击连续运行按钮,测量不同处的温度(例如空气和人体等),
改变温度值单位,观察程序运行情况.之后停止并保存相关的文件.

\subsection{创建一个电压输出和采集的程序}

(1)新建空白 VI. 操作程序框图窗口.

(2)对于输入部分, 我们需要进行如下操作: 新建 "DAQmx 创建虚拟通道" 
(测量I/O $\to$ DAQmx -数据采集 $\to$ DAQmx 创建虚拟通道), 选择模拟输入电压. 
在其  "物理通道"  接口处右键单击, 创建 "输入控件". 新建 "DAQmx -数据采集", 
"DAQmx 读取", "DAQmx 清除任务", 并在 "DAQmx 读取" 的  "数据"  接口处右键单击, 
创建"显示控件". 新建"While 循环", "等待", 在 "等待" 的 "等待时间 ms" 接口处右键单击, 
创建 "常量", 设为 $100$, 在 "结束条件" 处右键单击, 创建停止. 调整位置, 连线, 更改标签.

(3)输出部分与之相似, 不同之处有: 选择:  "DAQmx 创建虚拟通道" 
选择模拟输入电压; 新建 "DAQmx 写入" 而非 "DAQmx 读取"; 
在 "DAQmx 写入" 的  "数据"  接口处右键单击, 创建 "输入控件".

这一部分讲义上只一句"用类似的方法创建电压采集程序",但其实有所不同,需要按照实际功能的要求选择.
我在这一部分也请教了老师才做出正确的程序图.
\begin{figure}[htbp]
    \centering
    \includegraphics[width=12cm]{电压输出与采集程序.JPG}
    \caption{实验二:电压输出和采集程序}
\end{figure}

(4)在前面板中, 则需要修改标签, 并调整图表位置. 
而对于实验所使用的 ELVIS 仪器面板, 只需要在打开两个电源后, 
将 AI 0+ 与 AO 0, AI 0- 与 AIGND 用导线连接即可. 
然后即可对输出通道/输入通道分别选择 Dev1/ao0 和 Dev1/ai0, 
改变输出电压, 观察测量电压(见下图). 最后保存并关闭文件.

\begin{figure}[htbp]
    \centering
    \includegraphics[width=12cm]{电压 运行.JPG}
    \caption{实验二:电压输出和采集程序的运行}
\end{figure}

\subsection{用虚拟仪器测量伏安特性}

(1)新建 VI 文件, 在前面板新建 "Express XY图", 
修改标签, 选用"点加线"模式, 将横坐标和纵坐标标签分别修改为 
"电流(A)" 和 "电压(V)" . 新建"数值输入控件" $\times 4$, 
修改标签并设置单位; 新建"数值显示控件", 修改标签; 新建 "开关按钮", 
"数值显示控件", "数组", 拖拽使数组成为 $2\times 20$ 以上大小, 
将 "数值显示控件"拖放至数组框内并拖拽成$2$个.

(2)编写程序框图的步骤繁琐, 不再赘述, 
值得提示和需要注意的是: while内的并行结构需要共计5个帧, 
右键边框添加即可; 第1帧 (注: 按照变成的习惯, 首帧为第0帧) 
中的陌生图标是 "进制转换"; 第2帧中的陌生图标是 "索引数组"; 
while外部上方和右方的陌生图标是 "创建数组"; 连接 "待测电阻值" 的陌生图标是 "线性拟合", 
后者使用输出端 "斜率"; while循环框上的图表为寄存器, 直接右键添加即可; 
所有 "DAQ助手" 的生成模式均选择"1 采样(按要求)" 即可, 
其中有一个并不做要求.

这部分程序框图的搭建犹其需要细心,功能和连线端口等选择容易出错.
\begin{figure}[htbp]
    \centering
    \includegraphics[width=17cm]{伏安特性.JPG}
    \caption{实验三:用虚拟仪器测量伏安特性}
\end{figure}

(3)接下来需要连接外部电路, 如图所示, 
连接好原型板上导线和电阻即可, 其中的蓝色电阻可更换. 
连接完之后即可运行程序, 每次测量只需要运行一次程序即可. 
在数组箭头处按右键可以将数据导出为 Excel, X-Y图按右键可以将数据导出为位图. 
在每次实验后更换电阻/稳压二极管(正反), 得到四项数据, 
导出并存储即可. 

% \begin{figure}[htbp]
%     \centering
%     \includegraphics[width=12cm]{qq_pic_merged_1698676578044.jpg}
%     \caption{实验三:用虚拟仪器测量伏安特性实际电路图}
% \end{figure}



\subsection{二线电流电压分析仪}
\subsection{基于口袋教学平台——NI myDAQ 便携式数据采集器的自行实验}

*这两部分并无要求,故无实验内容展示


\newpage
\section{实验结果与数据分析处理}


\begin{enumerate}

    \item 初步熟悉 LabVIEW 开发环境的基本操作和编程方法
    
    本部分仅需了解即可, 无运行结果与数据处理需求.
    
    \item 创建一个温度测量程序
    
    \begin{enumerate}
        
        \item 创建一个模拟温度测量程序
        
        点击按钮后, 程序持续运行. 在前面板修改采集的电压: 
        在 "采集的电压" 输入框输入 $0.5 \sim 2$ 之间的值, 可以获得不同的温度值. 
        
        显示数值与预测值相符, 在选择摄氏温度的情况下, 显示温度是电压的100倍; 
        选择摄氏度时, 需先$-32$再$\div 1.8$, 再输出结果.
        这符合我们的理论换算比例.
        
        \item 使用真实热电偶元件创立一个温度测量程序
        
        点击按钮后, 程序运行. 手握热电偶的一端, 会发现温度值与之前的有不同. 
        拔掉热电偶, 电阻也会有不同 (直接消失). 
      
    \end{enumerate}
    
    \item 创建一个电压输出和采集的程序
    
    点击按钮后, 程序持续运行. 输入一个电压, 
    程序随即会给出一个电压的测量值, 与输入值大小相同(见上述实验运行图,二者严格相等).
    
    点击"停止输出"/"停止测量"按钮后, 确实停止了测量电压的变化; 
     但是再多点击几次两个按钮时, 测量电压又会开始随输出电压发生变化.
    
    \item 利用虚拟仪器测量伏安特性(不同阻值电阻处理)
    
    输入好相应的参数, (不同待测电路元件对应的输入参数如表1)再点击运行按钮后, 程序运行一次, 
    并给出测量的电流/电压成对数据, 然后可计算出线性拟合的电阻值或者观察二极管的伏安特性曲线.
   
    % Table generated by Excel2LaTeX from sheet 'Sheet1'
    \begin{table}[htbp]
        \centering
        \caption{不同待测电路元件对应的输入参数}
        \begin{tabular}{|c|c|c|c|}
            \hline
                    & $1 \unit{k$\Omega$}$ 电阻  & $51 \unit{$\Omega$}$ 电阻  & 二极管 \bigstrut\\
            \hline
            输出电压步长 & 0.25    & 0.25   & 0.07 \bigstrut\\
            \hline
            测量数据点数 & \multicolumn{3}{c|}{20} \bigstrut\\
            \hline
            标准电阻   & \multicolumn{3}{c|}{100} \bigstrut\\
            \hline
            时间间隔   & \multicolumn{3}{c|}{0.02} \bigstrut\\
            \hline
        \end{tabular}%
    \end{table}%

    \newpage
    测量得到的51$\Omega $和1000$\Omega$电阻的电压-电流 ($U$-$I$) 
    结果如表2所示
    
    (原电压、电流单位为V和A,为方便处理,转换为mV和mA)
    

    
      \begin{table}[!ht]
        \centering
        \caption{测量不同电阻的伏安特性}
        \begin{tabular}{ccccc}
        \toprule
        \multirow{2}{*}{数据点} & \multicolumn{2}{c}{1000$\Omega$}&  \multicolumn{2}{c}{51$\Omega$}\\
        \cmidrule{2-3}\cmidrule{4-5}
               & 电流(mA) & 电压(mV) & 电流(mA) & 电压(mV) \\ \hline
        1 & 0.452858 & 452.744 & 1.63141 & 83.7222 \\ 
        2 & 0.90689 & 907.098 & 3.28009 & 167.122 \\ 
        3 & 1.36092 & 1360.81 & 4.91267 & 251.811 \\ 
        4 & 1.81495 & 1813.88 & 6.56136 & 335.533 \\ 
        5 & 2.27543 & 2267.59 & 8.19716 & 420.222 \\ 
        6 & 2.72624 & 2722.28 & 9.83941 & 503.623 \\ 
        7 & 3.18027 & 3176.64 & 11.4817 & 587.346 \\ 
        8 & 3.64074 & 3630.04 & 11.6845 & 597.65 \\ 
        9 & 4.09155 & 4084.08 & 11.5654 & 591.532 \\ 
        10 & 4.5488 & 4537.49 & 11.4978 & 588.312 \\ 
        11 & 5.00606 & 4991.86 & 11.443 & 586.38 \\ 
        12 & 5.46331 & 5446.24 & 11.4108 & 584.77 \\ 
        13 & 5.92056 & 5899.33 & 11.3883 & 583.16 \\ 
        14 & 6.37137 & 6354.36 & 11.3722 & 582.194 \\ 
        15 & 6.83185 & 6808.1 & 11.3593 & 581.228 \\ 
        16 & 7.2891 & 7260.89 & 11.3496 & 581.228 \\ 
        17 & 7.73991 & 7716.25 & 11.3432 & 579.94 \\ 
        18 & 8.2036 & 8169.04 & 11.3303 & 580.584 \\ 
        19 & 8.66086 & 8623.13 & 11.3271 & 579.94 \\ 
        20 & 9.11167 & 9077.55 & 11.3239 & 579.94 \\
            \bottomrule
        \end{tabular}
    \end{table}

其伏安特性图像分别为图9中的两个子图:
    \begin{figure}[H]
        \centering
        \subfigure[$51{\Omega}$ 电阻]{\includegraphics[width=8cm]{图51.jpg}}\hspace{0.5cm}
        \subfigure[$1000{\Omega}$ 电阻]{\includegraphics[width=8cm]{图1000.jpg}}
        \caption{不同阻值电阻测量得到的伏安曲线图}
    \end{figure}
    
    \newpage
    根据最小二乘法原理, 假设电压与电流满足线性关系 $U = RI + U_0$ ($U_0$ 为微小干扰值, 
    根据实验数据明显能看出 $U_0 \approx 0$) 
    我们可以根据以下公式计算出两个电阻的测量阻值: 
    \[
        \hat{R} = \cfrac{\sum\limits_{i=1}^{n} (I_i - \overline{I}) (U_i - \overline{U})}{\sum\limits_{i=1}^{n} (I_i - \overline{I})^2}
    \]

    对 $1\unit{k$\Omega$}$ 电阻和$51\unit{$\Omega$}$ 电阻计算出的阻值(图像中的数值精度不太高,考虑调用Excel函数:SLOPE,计算)
    分别为 $995.576504\unit{$\Omega$}$ 和 $51.19982518\unit{$\Omega$}$. 
    与真实的电阻值特别接近(参见下述表3). 
    
    而两组数据的相关系数计算公式为:
    \[
        \rho_{xy} = \cfrac{\operatorname{Cov (X,Y)}} {\sigma_{x}\sigma_{y}} =  \cfrac{\sum\limits_{i=1}^{n} (I_i - \overline{I}) (U_i - \overline{U})}{\sqrt{\sum\limits_{i=1}^{n} (I_i - \overline{I})^2} \sqrt{\sum\limits_{i=1}^{n} (U_i - \overline{U})^2}}
    \]
    
    图示两个图像的相关系数均为1,我们考虑更加精细的计算:

    对 $1\unit{k$\Omega$}$ 电阻和$51\unit{$\Omega$}$ 电阻计算出的相关系数(调用Excel的PEARSON函数)分别为 $0.999999295$ 和 $0.999995973$, 
    在考虑一定的不确定度后, 两个相关系数均不小于$0.9999$, 
    可见两组数据的线性性十分强, 测量足够准确.

    从图9中(a)图我们发现后14个数据测量点很密集,为了确定这一部分的实验数据精度,单独计算其斜率与相关系数(同样调用Excel的PEARSON函数和SLOPE函数),可知测量得阻值为
    49.26405653$\Omega$,相关系数为0.997068622(图10中相关系数$R^2$显示为0.9941,更精细的结果为0.994145837.这是因为线性拟合下,有关系:$R^2=\rho_{xy}^2$),精度还是挺高的(图像如下).

    
\begin{figure}[htbp]
    \centering
    \includegraphics[width=10cm]{图511.png}
    \caption{51Ω电阻的伏安特性曲线(后14个数据点)}
\end{figure}

我们汇总一下上述数据,得到表3,可见我们实验电路和虚拟仪器的搭建很成功:
    \begin{table}[!ht]
        \centering
        \caption{电阻伏安特性曲线精细计算}
        \begin{tabular}{cccc}
        \toprule
                  & 1000Ω & 51Ω & 51Ω(后14个数据) \\ \hline
            测量阻值$\hat R$ & 995.576504 & 51.19982518 & 49.26405653\\ 
            相对误差(\%) & 0.004423496 & 0.003918141 & 0.034038107 \\ 
            相关系数$\rho_{xy}$ & 0.999999295 & 0.999995973 & 0.997068622\\ 
        \bottomrule
        \end{tabular}
    \end{table}
    
    \newpage

\item 用虚拟仪器测量伏安特性(二极管的伏安特性曲线)

   这一部分实验并没有新的操作,只需要在实际电路中将电阻换为二极管即可.
   
   我记录到的实验数据如下表:
\begin{table}[!ht]
    \centering
    \caption{测量不同电阻的伏安特性}
    \begin{tabular}{ccccc}
    \toprule
    \multirow{2}{*}{数据点} & \multicolumn{2}{c}{反向}&  \multicolumn{2}{c}{正向}\\
    \cmidrule{2-3}\cmidrule{4-5}
           & 电压(mV) & 电流(mA) & 电压(mV) & 电流(mA) \\ \hline
           1 & -69.8758 & -0.0108339 & 68.2657 & -0.00439373 \\ 
           2 & -139.108 & -0.0108339 & 139.108 & -0.0108339 \\ 
           3 & -208.983 & -0.00761381 & 208.339 & -0.00761381 \\ 
           4 & -279.181 & -0.00761381 & 278.859 & -0.00439373 \\ 
           5 & -348.735 & -0.00761381 & 348.091 & 0.00204643 \\ 
           6 & -418.933 & -0.00439373 & 416.679 & 0.0084866 \\ 
           7 & -489.131 & -0.00761381 & 481.403 & 0.0664481 \\ 
           8 & -558.685 & -0.00761381 & 533.568 & 0.246773 \\ 
           9 & -629.205 & -0.00761381 & 572.531 & 0.552681 \\ 
           10 & -698.758 & -0.00761381 & 598.614 & 0.984172 \\ 
           11 & -769.278 & -0.0108339 & 618.256 & 1.48006 \\ 
           12 & -839.154 & -0.0108339 & 632.425 & 2.0307 \\ 
           13 & -909.674 & -0.00761381 & 644.339 & 2.61353 \\ 
           14 & -979.228 & -0.00761381 & 654.322 & 3.20925 \\ 
           15 & -1048.46 & -0.00439373 & 662.372 & 3.81785 \\ 
           16 & -1118.98 & -0.0108339 & 669.135 & 4.44254 \\ 
           17 & -1188.21 & -0.00439373 & 675.897 & 5.07046 \\ 
           18 & -1258.41 & -0.00439373 & 682.015 & 5.70482 \\ 
           19 & -1328.93 & -0.00761381 & 685.88 & 6.35205 \\ 
           20 & -1398.81 & -0.00761381 & 690.71 & 7.00251 \\ 
        \bottomrule
    \end{tabular}
\end{table}

  
   理论上,电压较小时,发光二极管电阻几乎是0,
   同时注意到发光二极管的正向导通电压约为0.6V左右.
   在达到导通电压前,图线几乎没有变化,
   非常平稳,而在达到导通电压后电流以指数级别迅速增长 

而反向未达到击穿电压时,理论上通过二极管的电流是少数载流子漂移
运动所形成反向电流(电阻很大,记得程序中显示阻值有将近50000$\Omega$!),很小.

这和实际测量到的数据所作的图一致:

\begin{figure}[H]
    \centering
    \includegraphics[width=13cm]{图er.png}
    \caption{二极管的伏安特性曲线}
\end{figure}

为了更加精确地判断数据的精度,
我们考察二极管正向部分曲线(这里取正向第九组数据,
即V=572.531 $mV$作为起始点拟合).
目前的文献大多采用指数函数拟合二极管的正向伏安特性曲线.
例如,一种理论推导指出,PN结两端所加电压和流过它的电流之间的关系为:
\[
  i=i_s({\mathrm e}^{\frac{qu}{kT}-1})
\]
其中,$i_s$为反向饱和电流,q为电子电量,k为玻尔兹曼常数,T为热力学温度.
我们考察Excel图像中的指数拟合,得到如下图像:

\begin{figure}[htbp]
    \centering
    \includegraphics[width=12cm]{图er1.png}
    \caption{二极管正向导通曲线拟合}
\end{figure}

可见相关系数为$R^2=0.9999$(更加精细的计算表明为0.9998851896),可见实测曲线与拟合曲线几乎没有区别,测量数据十分之精确.
进一步地,我们可以反推出
\[
   T=\frac{1.60\times 10^{-19}}{0.0214\times10^3\times 1.38\times10^{-23}}{\mathrm K}\approx 514.8\mathrm{K}
\]
说明二极管温度在一合理的量级(尽管存在误差),
这从侧面印证了估计的合理性.

\end{enumerate}
    

\section{思考题}

*\bf{这一部分回答讲义上和暴老师讲义ppt上的一些问题}

\subsection*{思考题1:  \textbf{虚拟仪器系统与传统仪器有什么区别?请简要说明.}}

相较于传统仪器而言,虚拟仪器有以下的优点:

(1)虚拟仪器的可以显示的数据形式更加丰富,而且借助于计算机,我们可以进行拟合等,使得实验更有价值.而且其可以很方便进行数据的转换与导出,为我们实验提供了很大的便利,实验效率大大提高.

(2)使用虚拟仪器的操作会更加自由,界面完全由实验者定义, 是统一的图形化编程界面.我们可以自己定义一些功能,甚至“造”出新的仪器出来.在可设计性方面虚拟仪器更好.

(3)虚拟仪器系统架构在与仪器分立的计算机上, 我们通过不同的编程手段, 将之与测量仪器进行沟通交流, 最终实现传统仪器的机械化使用.
虚拟仪器实际上实现的是对实际测量仪器的操控, 是作为顶部的操控者, 对底部的模块进行统一、有序的调动和管理, 便于我们获得实验数据. 

总之,虚拟仪器界面简便、精确度高、多功能化、可实现仪器的智能化.
但虚拟仪器相较于传统仪器,也有缺点:比如虚拟仪器也需要使用者有一定的编程功底, 要求更高.又如虚拟仪器需要实际仪器的辅助,受实际仪器的限制.

\subsection*{思考题2:\textbf{所谓“虚拟仪器”的“虚拟”二字,如何解释?}}
虚拟仪器这个概念是由National Instruments公司提出的,其被称为“虚拟”,是相对于传统仪器的“真实”而言的,其主要区别是:

(A)虚拟仪器的硬件部分是由基本的测试硬件模块和计算机组成,两者通过某种总线形式进行有机结合后构成基本的硬件平台,然后在计算机屏幕上显示虚拟的仪器面板.

(B)在虚拟面板上除数据结果的波形图显示外,也提供了各种如开关按钮等控件对仪器进行操作;而传统仪器的操作按钮是相对固定、功能确定的,并且操作的对象和方式均是实体.

(C)虚拟仪器的内部功能也是通过软件编程在计算机上实现的,而传统仪器需要基于其内部设定好的内部线路和硬件结构实现其特定功能.

\subsection*{思考题3:\textbf{我们平时实验中使用的仪器有什么特点和限制?}}
传统的独立仪器(如示波器、数字多用表、函数发生器等),特点是包含信号输入、分析处理、显示和控制几个部分.
我们通过界面按钮实现特定化的测量和数据显示,故缺点也很明显:
数据处理能力一般和显示形式比较单一,且不能很好共享数据与自动化、智能化等

% \subsection*{思考题4:\textbf{如果使用真实的电压表,能否绘制出测量电源电压和双电压表测待测电阻阻值的电路图?}}
% 如图所示:
% \begin{figure}[htbp]
%     \centering
%     \includegraphics[width=8cm]{图la.png}
% \end{figure}

\subsection*{思考题4:  \textbf{本实验内容3中的电压输出和采集哪个先执行?}}
实际上,作为一个自动并行化执行的G语言系统,LabVIEW 语言会自动进行多线程处理(并行处理),
电压输出部分和采集部分的数据流同时产生(因为我们并没有规定先后顺序),因而可以认为两个部分是同时开始执行的.
这可能与传统的编程语言 (基本部分) 不同, 但与一些硬件解释语言相类似. 其实 LabVIEW 平台也的确是在类似描述一个虚拟的 "硬件仪器", 并行执行是很合理的.

但是,如果从本次实验的实际应用上考虑,我们可以采用先输出后采集的方式,以避免开启电路时的缓冲对测量结果的干扰,使得测量更精准.

\newpage
\section{感想总结}
在这次实验中, 
我利用LabVIEW 2014平台, 
设计了四个电路, 了解了虚拟仪器相关的知识, 
初步认识了这种图形化编程软件, 感受了其与实际电路连接的方法, 
体验了热电偶的热电动势信号如何传递到电脑当中, 
并绘制了电阻与二极管的伏安特性曲线.
    
在本次实验之后, 
我对LabVIEW开发环境和图形化语言G的编程略知一二, 
知道了其中的大体部分, 
学会了一些 "函数" 的使用.
 G语言是一个非常图形化的语言, 
 编程的基础部分与C语言比较相近, 但其实又有所不同:
 一是它不需要编写枯燥的程序代码, 
 但是需要通过图形去连接不同部分; 
 二是它有两个面板, 前面板和后面板, 
 编程需要两个部分的配合,这和普通的编程还是有不少区别的.
 让我在编写程序的过程中遇到了一些困难, 
 比如连线复杂/寻找工具困难和连线错位之类.不过幸好及时请教暴老师,实验也顺利高效地完成.

电脑中的程序是虚拟的部分, 
原型板上的电路是实际的部分, 
虚拟和现实在虚拟仪器中结合在了一起. 

以后如果需要测量一些物理量, 
我觉得使用LabVIEW 2014进行编程可能可以省去很多时间. 
比如说在之前预科实验测量二极管性质的时候, 
说不定可以使用这个仪器进行测量. 
像这种进行大量测量的物理实验, 
大量简单重复的实验步骤可以由机器代替我们人类进行; 
而人类需要做到的就是利用机器帮助人们减少工作量.

在撰写本次实验报告的过程中,我新配置了\LaTeX 环境,学习了如何Excel处理数据中的一些特殊函数(比如此次用到的PEARSON函数、SLOPE函数和LINEST函数等)
以及数据的常规处理和分析,这无疑对我之后的物理实验报告撰写乃至科研生活都是有益的.
此外,本次实验鲜明地体现了物理需要"理论"同"实验"结合的学科特色.正如荀子所言:“吾尝终日而思矣,不如须臾之所学也”.物理知识的“学习”也是如此,如果只是
从书本上看,从课堂上听,而没有动手亲自“学”的话,终究对知识的理解和把握是有限的.物理本身
就是一门以实验为基础的学科,理论与实验的结合很重要:在做实验时需要利用理论对实验过程与
方法不断进行重新设计与思考;反过来做理论也要结合实验现象给出进一步的预测,二者可以说是
一个相辅相成的关系.

本次实验的成功高效地完成,离不开实验前的预习、以及实验过程中暴老师不厌其烦的答疑.虚拟仪器技术已成为自动化测量的主流技术,
我也有幸在本次实验中体会到了这一点.

————————

(此次实验并没有相应的预习报告提交要求和实验数据记录表,故无附件)


\end{document}